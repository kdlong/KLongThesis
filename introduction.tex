\chapter{Introduction}
\label{ch:introduction}

\section{The universe of particles}

The idea of understanding the physical world by identifying its
fundamental constituents is a very old one. Ancient eastern and
western philosophers independently attempted the feat, concluding
that the world could be reduced 
to ``elements'' such as water, earth, and fire.
As observational science became a more prominent component of modern thought, 
evidence mounted that these elements themselves were 
not indivisible. Perhaps they had not identified
the correct fundamental elements, but the belief that this endeavor could
be accomplished lived on.

A major attraction of identifying the fundamental building blocks of the universe
is the implication for constructing a macroscopic description from these
pieces. This process of understanding a complex system 
by reducing it to its essential components,
termed ``reductionism'' by philosophers, 
has formed the backbone of scientific endeavor for much of history.
In practice, the step from the fundamental to the macroscopic 
is far from trivial. Despite a well-established
understanding of the proton constituents and their interactions,
predicting the proton mass from first principles has only recently 
been achieved~\cite{Durr:2008zz}, 
and larger systems remain well out of reach.
In addition to the challenges of calculability, additional structures
may arise at the macroscopic level that require unique observations
about the nature of large systems, as is the case
superconductivity~\cite{Anderson393}.
It may never be feasible
to achieve a theory of biology built from the findings of particle 
physics, but there is an undeniable elegance in achieving a concise description
of the fundamental elements of the natural world and their interactions,
and this remains the target of the field of particle physics today. 

For over a century, particle physics has been characterized by experimental
discoveries, many predicted by theoretical inferences, and others 
completely unexpected.
They have dramatically propelled the field forward,
while at times upending the accepted thought.
The history of particle physics is a fascinating one, with a subtle 
and intricate narrative justifying a much deeper discussion than presented
here. Ref.~\cite{Griffiths:2008zz} gives a technical introduction to the field in a historical
context, whereas Ref.~\cite{kragh2002quantum} gives a purely 
historical account. An encyclopedic overview of the major
theoretical and experimental advances in particle physics is given in Ref.~\cite{Ezhela:1996xi}.

The modern genesis of the field can reasonably be traced to 
the discovery of the electron in 1897 by J. J. Thomson~\cite{doi:10.1080/14786449708621070}, a particle still
believed to be elementary today.
At the time, though supported by the thermodynamic studies of Boltzman,
the once-popular idea of a quantized unit of matter was
held in widespread doubt. 
The investigation of radioactive decays, by Becquerel,
Marie and Pierre Curie, Rutherford, and others~\cite{RADVANYI2017544}, 
became closely linked with efforts to categorize
the structure of the chemical elements, assumed to be
building blocks of nature. 
However, as the atomic model became accepted, 
it quickly became clear that the atom itself is not elementary.
At the turn of the century, the electron was incorporated into 
new models of a composite atom.
The work of Rutherford demonstrated the presence of the
positively charged nucleus at the center of the atom~\cite{Rutherford:1911zz}.
The positively-charged proton, with a mass nearly two thousand times
that of the electron, was discovered in 1919, again by Rutherford,
which he correctly associated with the 
nucleus~\cite{doi:10.1080/14786440608635919}.

A mathematical understanding of the electromagnetic interactions 
governing the proton and electron had
already been established by James Clerk Maxwell decades before their discoveries.
Combined with the new quantum theory, rapidly developing in the 1920s and 1930s, the electromagnetic
interaction is also sufficient to explain the stability and arrangements of
electrons into atomic orbitals around the nucleus. 
A key aspect of this model is the fact that the electron, and all other
fundamental particles, carry a fundamental
angular momentum and magnetic moment, termed ``spin.'' 
As required by the quantum theory, and 
demonstrated by Stern and Gerlach in 1922~\cite{1922ZPhy....9..349G},
spin is quantized.
The electron and proton spins are always observed to be $\pm\hbar/2$, where 
$\hbar=1.055\times 10^{-34}\unit{J}\cdot\mathrm{s}$
is the reduced Planck constant.
We refer to such half-integer spin particles as \emph{fermions}. Particles
with scalar spin are referred to as \emph{bosons}. In particular, spin-1/2 objects
are known as spinors, whereas spin-1 objects are vectors.

A major success of the quantum theory is the resolution of an apparent
conflict between the wave-like and particle-like nature of light:
in relativistic quantum mechanics (quantum field theory, or QFT), all particles are associated with fields, giving
them both wave-like and particle-like properties. 
By the early 1940s, a relativistic quantum theory of the charged particles
and their interactions via the electromagnetic force had been developed in 
this language. 
Quantum electrodynamics (QED) is a QFT in which interactions 
follow from the invariance of a spin-1 field under a class of transformation of the field properties
known as \emph{gauge transformations}.
In such a quantum gauge theory, the \empf{gauge symmetry} of the spin-1 field leads to interactions between the fermions
a boson fields.
In the theory of quantum electrodynamics (QED),
electromagnetic interactions 
between fermions are communicated by the 
exchange of the spin-1 photon ($\gamma$),
the particle of light. After overcoming initial mathematical challenges, 
the predictive power of QED was astounding.

Yet problems with the atomic model remained. The mass of the atom
suggested another, neutral component of the nucleus. 
The neutron, with nearly equal mass to the proton, was
discovered by Chadwick in 1932~\cite{Chadwick:1932ma,doi:10.1098/rspa.1933.0152}. A new force, the ``strong force'' was proposed as
a means to 
confine the neutral and positively charged constituents of the nucleus,
but its mechanism was completely unknown.
Further confusion arose from atomic decays. Whereas many atomic
decays are consistent with a breakdown of the original atom into 
smaller atomic pieces, $\beta$ decays,
in which a radioactive element with atomic number $A$ 
(the number of protons) decays to an element with $A+1$
via the emission of an electron, are not. 

The first predictive theoretical treatment of $\beta$ decay 
was developed by Fermi in 1934~\cite{1934ZPhy...88..161F}. 
At a time when the structure
of the nucleus was still not convincingly established,
he recognized that all $\beta$ decays could be generalized to the process
$n\to\Pp+\mathrm{e}$ (or $n\to\Pp\mathrm{e}$ in a shorthand notation).
Because the energy distribution of electrons in $\beta$ decay is not constant,
a two-body decay is insufficient. Fermi adopted a hypothesis of Pauli
that a third particle, of very low mass and uncharged, also participated
in the interaction, 
He proposed a direct four-fermion interaction in which the
fermions could be created or destroyed,
motivated by photon emission in atomic transitions.
His unconventional proposals proved correct;
the neutrino was discovered in 1942 via the inverse process of $\beta$ decay~\cite{Cowan:1992xc}.
While his theory must be extended
to describe higher energy interactions (and expanded to consider the parity
violations, discussed in Section~\ref{sec:particles}), it is a low-energy limit of the 
modern theory of the weak interaction. The connection of theories that are valid
only for low-energy interactions to more complete theories has important implications for experimental
inferences, which are increasingly challenges at higher energies.
The techniques used in this thesis to study interactions beyond the directly accessible
energy scale are heavily influenced by this work.

Such low-energy versions of a more complete theory
are a valuable calculation al tool, and give insight to the high-energy theory. 

New equipment, such as the cloud chamber developed by Charles Wilson
in 1911~\cite{doi:10.1098/rspa.1912.0081},
enabled extensive characterization of radioactive decays.
In addition, it
opened our eyes to the radiation bombarding the earth from
outside the atmosphere. The discoveries of the positron in 1932~\cite{Anderson:1932zz,Anderson:1933mb}
and the muon in 1936~\cite{Street:1937me} by Anderson and others were made from such studies.
Both had important implications for the burgeoning field of particle
physics. The positron, a mirror image of the electron but with 
positive charge, had been predicted a year earlier by Dirac 
in his relativistic theory of the fermions~\cite{Dirac:1928hu}. 
His proposal of \emph{antiparticle} partners to the observed matter particles,
with equal mass but opposite quantum numbers,
was too outlandish for many to take seriously but proved prescient.
Further studies demonstrated that the muon behaves like a heavier
partner to the electron, the first sign of \emph{generations} of fermions.
Because the particle seemingly played no role
in the structure of the atomic, the existence of the muon was completely unexpected. 
Given the new breadth of the seemingly-fundamental particles
observed, not all closely associated with the atomic theory,
the field of particle physics was fully its own domain
by the 1940s, with the purpose of characterizing the fundamental objects of the
universe, their properties, and interactions.

\section{Particle scattering experiments}

In addition to studying the products of radioactive decays themselves, 
Earnest Rutherford realized the energetic decay products of
radioactive nuclei could 
themselves be used as probes of other nuclei.
By directing the 
helium nuclei (referred to as $\alpha$ particles) emitted from radioactive radium decay towards gold foil
and measuring the deflection
angles, he and his collaborators demonstrated that the deflection
patterns of the scattered $\alpha$ particles were consistent with 
a concentrated positive charge at the center of atoms,
the atomic nucleus~\cite{Rutherford:1911zz}.
Inferring information about the properties and interactions of particles based on their
behaviour when scattered---referred to as a ``scattering experiment''---is still central 
to particle physics experiments today.

This thesis concerns measurements of proton-proton (\pp) scattering. 
A study of scattering experiments in which protons colliding
at sufficiently high energy to overcome the electromagnetic repelling force 
to allow a direct interaction of the quarks and gluons that compose the protons is performed.
Fig.~\ref{fig:scattering} gives a sketch of the scattering experiment approach
to an important process in $\pp$ collisions, the details of which are not important: 
the production of the Higgs particle in decays to two photons ($\gamma$). 
Protons of known energy and momentum are brought to collisions and the
outgoing particles---photons, in this example---are detected. Measurements 
of the outgoing particles (the photons), along with knowledge of the incoming particles, are used to infer
information about the unseen interaction. As discussed in Chapters~\ref{ch:phenomenology}
and \ref{ch:simulation}, such diagrams are not just useful visualizations of the 
interaction (in this case, the production and decay of the Higgs boson); they can be
connected to mathematical expressions used to predict explicit properties of
the outgoing particles. In this example, predictions of photon properties both with
and without the existence of the Higgs boson can be compared to the experimental observation,
which conclusively favors the with-Higgs-boson scenario.
In collisions of interest to this work, the quark and gluon interactions produce massive 
vector bosons of the weak force that themselves interact. The consistency of these 
events with their expected properties is measured and used to test theories 
of new particles or new interactions.

\begin{figure}[htbp]
  \centering
   \includegraphics[width=0.9\textwidth]{figures/Chapter1/ScatteringExperiment.png}
  \caption{
    Illustration of Higgs boson production with decay to two photons 
    in a $\pp$ scattering experiment.
  }
 \label{fig:scattering}
\end{figure}

Because high-energy particle interactions are driven by quantum mechanics, they are stochastic
in nature. Therefore, a single collision cannot be uniquely interpreted in terms
of a given interaction (or ``process''). Rather, features of a measured process,
such as the consistency or inconsistency with a hypothetical interaction or the presence
of a new particle, must be inferred from collective results.
To this end, a critical metric of scattering processes is the scattering cross section.
For a particle beam scattered off a target, or equivalently, two beams of particles
brought to collision, the cross section is the effective area of the beam
deflected by scattering from the target (or other beam). For two beams with 
densities $\rho_{\mathcal{A}}(x)$, $\rho_{\mathcal{B}}(x)$ and length $\ell_{\mathcal{A}}$, 
$\ell_{\mathcal{B}}$, the cross section is defined as~\cite{Peskin:1995ev}
\begin{equation}
  \sigma = \frac{\text{Number of scattering events}}
    {\ell_\mathcal{A}\ell_\mathcal{B}\int\mathrm{d}^2x\rho_{\mathcal{A}}(x) \rho_{\mathcal{B}}(x)}\,.
  \label{eq:crossSection}
\end{equation}
For example, the cross section $\sigma_{\pp\to\EE}$ relates the number of 
events with two electrons produced in the scattering of two proton beams.
The cross section is only unambiguously defined experimentally when it relates
only the initial state and final states. 
However, it is common to discuss the cross section
for, e.g., Higgs boson production, under the assumption that a theoretical extrapolation
has been made in order to connect the measured particles (e.g., photons) to their production
mechanism.
Cross sections are expressed in units of area. It is customary to use the 
units of barns (b), where $1\unit{b} = 10^{-24}\cm^{2}$. 

It is also often useful to discuss not just the total cross section for all scattering to a final
state, but to consider a \emph{differential cross section}, dependent on properties
of the final state particles. For example, the differential cross section 
$\mathrm{d}\sigma_{\pp\to\EE}/\mathrm{d}E_{\Pem}$ is measured by dividing 
the number of scattering events by properties of the outgoing electron energy $E_{\Pem}$
to produce a differential distribution of the production rate in terms of $E_{\Pem}$.
A measurement may also be restricted to consider only the scattering of outgoing particles
into a specific geometrical region or above some threshold energy and momentum. Such a measurement
is termed a \emph{fiducial cross section}, restricted to a fiducial phase space
defined by the geometrical and kinematic restrictions of the measurement.

\section{Particle colliders and theoretical foundations}
Rutherford continued to use the radioactive decays of elements as a means
for particle scattering measurements through the early 1900s. Yet it was clear that
a means of accelerating particles to collision at higher energies would be needed
to further probe the structure of the nucleus.
This requirement lead to the invention of
the first particle accelerators, which have come to define the field ever since.
An electrostatic accelerator was developed at the Cavendish Laboratory
in Cambridge, England by John Crockcroft and Ernest Walton and
was used to split Lithium into Helium atoms in 1932~\cite{CrockcroftWalton}.
Concurrently in California, Ernest Lawrence developed the cyclotron to 
accelerate charged particles along a helical trajectory
through electric and magnetic fields~\cite{PhysRev.40.19},
which became a model for many future accelerator facilities.
These inventions triggered an explosion in the field, marked by a continual 
effort to collide particles at higher and higher energies, capable of exploring
the structure of matter and increasingly smaller distance scales.

Accelerators provided a new means to produce particles in the laboratory.
Combined with improved tools for detection, they introduced a new era
of exploration.
By the 1950s, particle physicists were faced with a ``particle zoo''
of seemingly fundamental objects spanning a huge range of masses. 
An underlying structure of the properties of many of these particles, 
now known as hadrons, was described by Gell-Man and Zweig in the early 1960s.
They realized the properties of the many observed hadrons---including the 
proton and neutron---could be 
attributed to combinations of different elementary particles,
which Gell-Man termed quarks.
Whether these particles were physical or abstract entities was 
not immediately evident, but experiments at the Stanford Linear Accelerator
a few years later showed that the proton has point-like constituents
with the properties Gell-Man and Zweig predicted~\cite{Tannenbaum:2018ogd}.

It remained unclear that the language of a quantum gauge theory, 
so successful in the formulation of the electromagnetic interaction, could 
also describe the strong or weak interactions. Both have crucial differences with respect to
the electromagnetic interaction: weak interactions extend over a much smaller
range, and the strong interactions has an inverted distance scaling, 
increasing at higher separation
of the interacting objects while decreasing at very short ranges. 
A quantum gauge theory of the weak interactions was achieved in the 1960s with leading
contributions from Glashow, Salam, and Weinberg~\cite{Weinberg:1967tq}.
Their theory unified the weak and electrodynamic interactions, with each communicated
by the exchange of vector bosons.
A crucial component of the electroweak theory is the presence of a scalar field
that allows the vector bosons of the weak force, the $\Wpm$ and $\PZ$ bosons,
to acquire mass through electroweak symmetry breaking, further discussed in Chapter~\ref{ch:phenomenology}.

A description of the strong force as a quantum gauge theory 
(termed quantum chromodynamics, or QCD) followed,
based on a more complex set of gauge fields.
David Gross, David Politzer and Frank Wilczek
overcame its computational complexity to demonstrate the required
inverse distance scaling properties of the theory in 1973~\cite{Gross:1973id,Politzer:1973fx}. 

Many major discoveries that enabled and confirmed these theoretical breakthroughs
are closely linked with technological advancements and collaborative
efforts leading to more powerful accelerators, including
\begin{itemize}
  \item The gluon, the vector boson of the QCD theory, discovered
      at the PETRA collider at Deutsches Elektronen-Synchrotron by the TASSO experiment~\cite{Brandelik:1979bd}, which provided
      conclusive support of QCD as a quantum gauge theory.
  \item The observations of the $\Wpm$ and $\PZ$ bosons, the massive vector bosons of the weak interaction,
    discovered by the UA1 experiment at the SPS collider at the European Center for Nuclear Research 
    (CERN)~\cite{Arnison:1983rp,Banner:1983jy,Arnison:1983mk,Bagnaia:1983zx}.
  \item The discovery of the top quark, the heaviest fundamental particle, by the D0 and CDF Collaborations
    at the Tevatron collider at Fermilab~\cite{D0:1995jca,Abe:1995hr}.
\end{itemize}
The most powerful collider ever built, the Large Hadron Collider (LHC) at CERN,
now allows the study of the most energetic particle collisions ever
recorded in the laboratory, reaching concentrated energies only replicated
in the most cataclysmic events in the universe. The LHC
was planned, developed, and built over the course of decades, with
personnel and funding from countries throughout the world.
It enabled the discovery of the Higgs boson, the most recent fundamental discovery in particle physics.
Its discovery in 2012 by the 
ATLAS~\cite{Aad:2012tfa} and CMS~\cite{Chatrchyan:2012xdj,Chatrchyan:2013lba} Collaborations
is the last piece in the EW theory, confirmed almost 50 years after its original proposal
by Higgs, Brout, Englert, Kibbel,
Hagen, and Guralnik~\cite{PhysRevLett.13.321,Higgs:1964ia,PhysRevLett.13.508,PhysRevLett.13.585,PhysRev.145.1156,PhysRev.155.1554}.

The LHC accelerates proton beams to energies of 
$6.5\times10^{12}$ terra-electronvolts (TeV),
where $1\unit{eV}=10^{-12}\TeV=1.602\times10^{-19}\unit{J}$, that are brought to collisions at 
a center-of-mass energy $\sqrt{s}=13\TeV$.
In addition to accelerating protons to groundbreaking energy, the LHC
delivers a very high rate of \pp collisions. This allows very rare 
interactions to be studied,
and is critical to obtaining statistically significant results.
In modern particle scattering experiments, the beam is not continuous,
as suggested by Equation~\ref{eq:crossSection}. It is helpful
to express the properties of the beam particle density and effective length 
in terms of the luminosity $\mathcal{L}$ which captures the geometric properties of
the beam such that
\begin{equation}
  \text{Number of scattering events} = \sigma\mathcal{L}\,.
\end{equation}
Luminosity has units of inverse area, customarily expressed in b$^{-1}$.

This work is based on studies of $\pp$ collisions delivered by the LHC 
and collected by the Compact Muon Solenoid detector in 2016, which 
are discussed in detail in Chapter~\ref{ch:lhcAndCMS} of this thesis.
The precise definition of the luminosity for this experiment will also be
discussed.

\section{The particle content of the standard model}
\label{sec:particles}

The experimental and theoretical work discussed here,
as well as many other successful discoveries, measurements, and 
breakthroughs---combined with countless null results and
unfulfilled hypotheses---have lead to the modestly-named 
standard model (SM) of particle physics.
With the discovery of the Higgs boson, the SM is widely considered to 
be complete~\cite{Quigg:2009vq}. 

The particle content of the SM is depicted in Fig.~\ref{fig:theparticles},
subdivided by the properties and quantum numbers of each particle.
Matter in the SM is composed of fermions with half integer spin; all fermions
in the SM have spin 1/2. The spin 1 vector bosons, highlighted in red,
mediate interactions between the fermions. Vector bosons are associated
with the SM forces they mediate: the electroweak force consists of
the electromagnetic force, mediated by the photon ($\gamma$), a neutral
component of the weak force, mediated by the $\PZ$ boson, a charged
component of the weak force mediated by the $\PW^{+}$ and $\PW^{-}$ bosons; the
strong force is mediated by the gluon ($\Pg$).

The particle masses, one of the most familiar properties, are also shown.
The units of mass, GeV/$c^{2}$, are motivated by the mass-energy relation
$E=mc^2$; 1 GeV/$c^2 = 1.79\times 10^{-27}\unit{kg}$. 
In this thesis, natural units (or God-given units, in the language of Ref.~\cite{Peskin:1995ev})
where $c=\hbar=1$ are used. Here $c$ is the speed of light, and $\hbar$ is the
Planck constant.
In this paradigm, the units of eV are used to express
mass, energy, and momentum.
Mass determines how particles interact with gravity, of which
a quantum description has not yet been achieved. 
The description of gravity is outside the scope of the SM, and
the gravitational forces on all particles produced at the LHC are
many orders of magnitude lower than even the weak interaction and are not considered.
The particle masses, which span orders of magnitude and do not follow a
conclusive pattern, are not predicted by the SM but must be experimentally
established. Predictive power with respect to particle masses would be a highly
desirable feature of an extension to the SM.

Each particle in the SM has an associated antiparticle, with the same
mass but opposite quantum numbers. 
Antiparticles are represented by explicitly specifying the charge, e.g. $e^{+}$ for the positron,
or with an overbar, such as $\Paq$ for a generic antiquark.
It is possible for a particle to be
its own antiparticle, as is the case with the photon and $\PZ$ boson,
and possibly the neutrino~\cite{Balantekin:2018azf}.
Particles and antiparticles are otherwise expected to have equivalent
properties, which has been experimentally validated, even for bound states 
such as antihydrogen~\cite{Ahmadi:2016fir}.
The predominance of matter over antimatter in the universe is one of the most profound
open questions in physics~\cite{Canetti:2012zc}.

\begin{figure}[htbp]
  \centering
   \includegraphics[width=0.9\textwidth]{figures/Chapter1/ChartOfParticles.png}
  \caption{
    The experimentally established particle content of the universe.
  }
 \label{fig:theparticles}
\end{figure}

The fermions can be
subdivided into ``quarks,'' highlighted in green in Fig.~\ref{fig:theparticles},
and ``leptons,'' highlighted in purple, depending on whether they carry the charge
associated with the strong force, known as ``color.''
There are three generations of fermions, divided by columns in
Fig.~\ref{fig:theparticles}. The first generation consists of regular matter:
the up-quark ($\cPqu$), down quark ($\cPqd$), and electron (e$^{-}$).
Second and third generation particles have similar properties
but larger mass: the muon ($\mu$) and tau ($\tau$) leptons, and the
charm ($\cPqc$) and strange ($\cPqs$) and top ($\cPqt$) and bottom ($\cPqb$) quarks.
Lepton number, of which leptons (antileptons) carry $L=1$ ($L=-1$) is conserved in the
SM, as is baryon number---defined analogously in terms of the number of 
quarks ($n_{\cPq}$) and antiquarks ($n_{\Paq}$) as $B=(n_{\cPq} - n_{\Paq})/3$.

The vector bosons couple to the fermions that are charged under the force
they mediate. All leptons but the neutrinos carry electric charge $-e$,
where $e = 1.602 \times 10^{-19}\unit{C}$.
Therefore, they couple to the photon and participate in the electromagnetic force. 
Quarks carry electric charge $+2/3e$ or $-1/3e$. The $+2/3e$ quarks, shown in the 
top row of Fig.~\ref{fig:theparticles}, are called up type, whereas the charge $-1/3e$
quarks are called down type (second row). 
Because the photon is massless, the range of the electromagnetic
force is infinite. Its strength decreases with the separation of the interacting particles.

Unlike leptons, quarks carry color charge. Therefore, they 
couple to the gluon and participate in the strong force. Like the photon, the gluon is massless,
yet unlike the electromagnetic force,
the strength of the strong interaction increases with distance. Consequently, the quarks form bound 
states of short distance pairs or triplets of quarks, bound together by gluon exchange.
There are three types of color charge: red, blue, and green (and respective
anticolor, carried by antiparticles).
Bound states of quarks are always ``colorless,'' either consisting of three quarks
each of a different color, or a color-anticolor pair.
We refer to $\Pq'\Paq$ bound states as mesons, the most familiar being the 
pions, formed from pairs of the $\cPqu$ and $\cPqd$ quarks.
The $\cPq\cPq'\cPq''$ states are known as baryons. They include the familiar proton, $\cPqu\cPqu\cPqd$,
and neutron, $\cPqu\cPqd\cPqd$.

All SM fermions participate in the weak interaction. 
As its name suggests, the weak force is significant only over very short distance scales,
a consequence of the fact that the $\PW$ and $\PZ$ bosons are massive.
Two separate charges
are associated with the weak force, weak hypercharge and weak isospin.
Because the $\PZ$ boson is uncharged, it communicates weak interactions of particles in which the
quantum numbers are unchanged. Conversely, the $\Wpm$ bosons carry weak 
hypercharge and weak isospin. Interactions mediated by the $\Wpm$ bosons do not conserve quark
flavor, that is, an interaction of quarks from one generation
may give rise to quarks of another generation. The degree of quark mixing between generations
is governed by the Cabibbo--Kobayashi--Maskawa (CKM) matrix, which must be determined experimentally.
A striking feature of the charged weak interaction is the violation of parity,
or mirror symmetry, first observed experimentally by Wu~\cite{Wu:1957my} by
measuring the $\beta$ decays of cobalt-60 (based on a proposal by Lee and Yang~\cite{Lee:1956qn}).
As spin 1/2 particles, fermions have two possible possible \emph{chirality} 
states, an abstract quantity that is equivalent to the particle helicity , or the 
relative alignment of the particle momentum and spin vectors, in the case of massless particles.
Only left-handed fermions and right-handed antifermions participate in the 
interactions mediated by the $\PW$ boson, which leads to the characteristic
parity violation.

The Higgs particle, shown in yellow,
has a unique role in the SM. It arises as a consequence of the spin-0 Higgs field, 
the only fundamental scalar in the SM.
The presence of the Higgs field leads to the {\PW} and {\PZ} bosons obtaining 
mass, through a process known as electroweak symmetry breaking (EWSB),
and experimental evidence suggests that the Higgs field is also responsible for the masses of the
quarks and leptons~\cite{Tanabashi:2018oca}.
Understanding the nature of the Higgs bosons and the scalar sector of the SM
is a major focus of particle physics today.

The underlying mathematical framework used to understand the quantum 
properties of these particles and their interactions, as well as an explicit 
discussion of EWSB, will be presented in Chapter~\ref{ch:phenomenology}. 

\section{Overview of this work}
This thesis presents measurements of particle production
in $\pp$ collisions at the CERN LHC.
Collisions giving rise to 
a total of three electrons or muons,
two forward ``jets'' of clustered hadronic particles,
and an imbalance of transverse momentum---characteristic of undetected neutrinos---are selected to 
isolate contributions involving the simultaneous
production of a $\PW^{+}$ or $\PW^{-}$ and a $\PZ$ boson, the heavy particles
that communicate the weak force.
Selected collisions of interest, referred to as events, are used
to measure the rate of production of processes predicted by the standard
model (SM) of particle physics, the most complete theoretical expression
of the known particles and forces of the universe, and to search for hypothetical extensions
of the SM. 

A dedicated search
for a rare, previously unobserved SM process, referred to as electroweak (EW)
WZ production (or \EWWZ production), is presented. 
This class of processes includes direct interact interactions of the $\Wpm$ and $\PZ$
bosons, which are precisely predicted by the SM.
The presence and production rate of the \EWWZ process is intimately connected to the 
mathematical structure of the weak interactions
of the massive vector bosons,
as well as the phenomenon of EW symmetry breaking (EWSB)~\cite{Quigg:2009vq}.

The discovery of a scalar boson with couplings consistent with those of the SM
Higgs boson by the ATLAS and CMS Collaborations~\cite{Aad:2012tfa,Chatrchyan:2012xdj,Chatrchyan:2013lba} 
provides evidence that the {\PW} and {\cPZ} bosons acquire mass through the
Brout-Englert-Higgs mechanism~\cite{PhysRevLett.13.321,Higgs:1964ia,PhysRevLett.13.508,PhysRevLett.13.585,PhysRev.145.1156,PhysRev.155.1554}.
However, current measurements of the Higgs boson 
couplings~\cite{Khachatryan:2016vau,Sirunyan:2018koj}
do not preclude the existence of scalar
isospin doublets, triplets, or higher isospin representations
alongside the single isospin doublet field
responsible for breaking the EW symmetry in the SM~\cite{Chiang:2018cgb,Chowdhury:2017aav}.
In addition to their couplings to the Higgs boson, 
the non-Abelian nature of the EW sector of the SM leads to 
quartic and triple self-interactions of the massive vector bosons.
New interactions or particles not predicted by the SM 
(generically referred to as physics ``beyond the SM,'' or BSM physics) 
in the EW sector is expected to include
interactions with the vector and Higgs bosons that modify their effective couplings. 
Characterizing the self-interactions of the
vector bosons is thus of great importance.

This thesis presents the first study of \EWWZ production performed by 
the CMS Collaboration~\cite{Sirunyan:2019ksz}.
Selected events are used to place constraints on BSM physics in terms
of explicit models predicting charged Higgs bosons, decaying to a $\Wpm$ and a $\PZ$
boson, and on generalized models of new interactions modifying
in the rate of production, or kinematic distributions, of selected events.

The outline of this work is as follows: Chapter~2 presents an extended 
overview of the theoretical underpinnings of this work, including the foundations
of the SM and the role of \EWWZ production in this framework. The motivations 
and structure of SM extensions probed in this work are also introduced.
Chapter~3 introduces the experimental setup and apparatus used to study W and 
Z boson production in the laboratory. The LHC and the Compact Muon Solenoid 
detector are presented and discussed. Chapter~4 describes the procedure of 
building predictions for vector boson production in pp collisions.
The use of these predictions in interpreting results, and in developing simulations
of particle production, decays, and interactions 
with the detector---used 
guide the analysis approach and optimizations---are discussed. Chapter~5 presents
the process of finding particle candidates from electronic signals in the detector.
Chapter~6 details the procedure of this analysis and the statistical underpinnings
used to extract results. Chapter~7 discusses the results obtained from this study
including interpretations and implications. Chapter~8 summarizes the
results presented and discusses future extensions.
