\chapter{Introduction}

This thesis presents measurements of W and Z vector boson production, 
decaying leptonically and associated
with two forward and high momentum jets, produced in proton-proton collisions
at the CERN LHC. The contribution to this process dominated by fully electroweak (EW)
interactions, which includes direct couplings of the W and Z bosons, 
is extracted and measured with an observed significance of
1.9 standard deviations, with 2.7 standard deviations expected. 
This is the first measurement of this process at CMS. 

The presence and production rate of the EW WZ process is intimately connected to the 
phenomenon of EW symmetry breaking (EWSB) \cite{Quigg:2009vq}, which gives rise to masses of
the vector bosons and their self-interactions. A fundamental prediction
of EWSB, fulfilled in the SM by the Brout-Englert-Higgs (BEH) mechanism,
is the emergence of a scalar particle, referred 
to as the Higgs boson. 
This particle was first observed by the 
ATLAS~\cite{Aad:2012tfa} and CMS~\cite{Chatrchyan:2012xdj,Chatrchyan:2013lba} Collaborations
at CERN in 2012, providing compelling evidence for the BEH mechanism as
a component of the SM theory.

Given the Higgs boson mass, the self-interactions 
of the W and Z bosons are precisely predicted in the SM. 
An excess of EW WZ events with respect to the SM prediction could indicate contributions from 
additional gauge boson or vector resonances~\cite{Delgado:2017cls}, 
additional Higgs or scalar bosons~\cite{Kilian:2015opv}, 
or it could suggest that the gauge or Higgs bosons are not elementary~\cite{Csaki:2015hcd}.
Measurements of this process can thus be used to constrain the predictions
of such beyond the SM (BSM) theories. 

This thesis presents constraints on BSM physics predicted by explicit models
including charged Higgs bosons 
and in the generalized framework of dimension-8 effective field theory (EFT) operators.
Constraints on the masses and couplings strengths of a hypothetical
charged Higgs boson presented here extend the previous CMS results~\cite{Sirunyan:2017sbn}, and
are complementary to those recently presented by the ATLAS Collaboration~\cite{Aaboud:2018ohp}. 
These results suggest that EWSB is uniquely fulfilled by the SM Higgs boson,
or that additional partners must be of significantly higher mass than the
SM Higgs boson.
Constraints on EFT operators provide limitations on a broad class of new physics
models, and can be interpreted in terms of a minimum energy scale at which additional
interactions than those predicted by the SM are feasible.
These results are the first constraints on dimension-8 EFT operators 
in from pp collisions with center of mass energy of 13 TeV in the WZ channel.

This study follows the first observations of EW diboson production
by the ATLAS~\cite{ATLAS-CONF-2018-030,ATLAS-CONF-2018-033} and CMS~\cite{Sirunyan:2017ret} Collaborations
and establishes EW WZ production as an important
probe of precision calculations in the SM while extending its importance 
in the search for New Physics.
All measurements presented are consistent with the standard model predictions,
providing support for the modern techniques for predictions concerning LHC collisions, and 
increasing evidence that the Higgs boson is the unique medium of EWSB 
at the energy scale probed by the 13 TeV LHC.

\section{The Universe of Particles}

The idea of understanding the physical world by identifying its
fundamental constituents is a very old one. Ancient eastern and
western thought attempted the feat, concluding
independently that the world could be reduced 
to ``elements'' such as water, earth, and fire.
As modern thought began to develop observational science in addition
to philosophy, evidence mounted that these elements themselves were 
not indivisible. Perhaps the Greeks had not identified
the correct fundamental elements, but the belief that this endeavor could
be accomplished lived on.

A major attraction of identifying the fundamental constituents 
is its implications for constructing a macroscopic description from these
pieces. This process of understanding a complex system through is pieces, 
termed ``Reductionism'' by philosophers, has enjoyed
many moments of triumph throughout history, 
In practice, the step from the fundamental to the macroscopic 
is far from trivial. Despite a well-established
understanding of the proton constituents and their interactions,
predicting the proton mass from first principles has only recently 
been achieved~\cite{Durr:2008zz}, and larger systems remain well
out of reach.
In addition to the challenges of calculability, additional structures
may arise from the nature of large systems, such as ferromagnetism or 
superconductivity.
This long-accepted path to a complete description of the physical world
has thus rightly been called into question, and it may never be feasible
to achieve a theory of biology built from the findings of particle 
physics~\cite{Anderson393}.
Yet there is an undeniable elegance in achieving a concise description
of the fundamental elements of the natural world and their interactions,
and this remains the target of the field of particle physics today. 

The modern genesis of the field can perhaps be credited to 
the discovery of the electron in 1897 by J. J. Thomson.
The electromagnetic interactions of this particle had been 
already been described by James Clerk Maxwell decades earlier.
The turn of the century began a saw an explosion in the theoretical
understanding of the 

\section{Particle Collider Experiments}

To observe the smallest pieces of matter, one must break matter into pieces.
The discovery of radioactivity in the late 1800s by Henri Becquerel had
profound implications for this idea, as it represented the spontaneous
disintegration of atoms into smaller pieces.

Studies of cosmic rays followed shortly afterwards, a


Be

Cosmic ray experiments
Transition from cosmic rays to colliders
Crockoff and Walton
Fixed target

\section{Motivation and context}

I would add that your first chapter
should also include some summary of your
goals for the analysis from the point
of view of why it is important to confirm
the SM predictions.

Your first chapter should also include
a brief summary of the experimental
results from at least CMS and ATLAS
preceding your thesis and the degree
to which improvement is important.

\section{Overview of results presented}

Here I give a brief overview of the thesis content.

