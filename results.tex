\chapter{Results}

\section{Fiducial WZjj cross section measurement}

The cross section for \WZjj production, without separating by production mechanism,
is measured with a combined maximum likelihood fit to the 
observed event yields by leptonic decay channel of the W and Z bosons as 
described in Section~\ref{sec:statistics}.
A signal strength $\mu_{\WZjj}$, which represents the 
ratio of the measured signal yield to the expected number of signal events, 
is treated as a free parameter in the fit.

\begin{figure}[htbp]
  \centering
   \includegraphics[width=0.7\textwidth]{figures/AnalysisResults/yieldByChannel.pdf}
  \caption{
    Post-fit event yields in the EW signal region.
          }
 \label{fig:EWSignalYields}
\end{figure}

The best fit value for the signal strength is used to obtain a cross section
in the tight fiducial region defined in Table~\ref{tab:selections}. 
The measured fiducial \WZjj cross section in this region is

\begin{equation}
  \sigma^{\mathrm{fid}}_{\mathrm{WZjj}} = 
        3.18^{+0.57}_{-0.52} \, \mathrm{(stat)} \,\, ^{+0.43}_{-0.36} \, \mathrm{(syst)}
        = 3.18^{+0.71}_{-0.63} \,\unit{fb} \,.
\end{equation}

This result can be compared to the predicted value of
$3.27 \, ^{+0.39}_{-0.32} \mathrm{(scale)} \pm 0.15\, \mathrm{(PDF)} \unit{fb}$.
The \EWWZ and \QCDWZ contributions are
calculated independently from the samples described in Section~\ref{sec:mc}
and their uncertainties are combined in quadrature. 
The interference term contribution in this region is less than 1\% of
the total cross section.

Results are also obtained in a looser fiducial region, defined in Table~\ref{tab:selections}
following Ref.~\cite{leshouches2017},
to simplify comparisons with theoretical calculations.
The acceptance from the loose to tight fiducial region
is $(72.4 \pm 0.8)\%$,
computed using \MG interfaced to \PYTHIA. 
The uncertainty in the acceptance is evaluated independently 
for the \EWWZ and \QCDWZ samples
from the scale
The uncertainty in the acceptance is evaluated 
by combining the scale and PDF uncertainties 
in the \EWWZ and \QCDWZ predictions in quadrature.
The scale uncertainty in the \QCDWZ contribution is the 
dominant component of the uncertainty.
The resulting \WZjj loose fiducial cross section is

\begin{equation}
  \sigma^{\mathrm{fid, loose}}_{\mathrm{WZjj}} = 
        4.39^{+0.78}_{-0.72} \, \mathrm{(stat)} \,\, ^{+0.60}_{-0.50} \, \mathrm{(syst)}
        = 4.39^{+0.98}_{-0.87} \,\unit{fb} \,,
\end{equation}

which can be compared to the predicted value of 
$4.51^{+0.59}_{-0.45} \, \mathrm{(scale)} \pm 0.18 \, \mathrm{(PDF)} \unit{fb}$.
The \EWWZ and \QCDWZ contributions 
and their uncertainties are treated independently with the same approach as described
for the tight fiducial region.

\section{Search for EW WZ boson production}

\begin{figure}[htbp]
  \centering
   \includegraphics[width=0.45\textwidth]{figures/AnalysisResults/mjj.pdf}
   \includegraphics[width=0.45\textwidth]{figures/AnalysisResults/dEtajj.pdf}
  \caption{
  The $\mjj$ (left) and $\left|\etajj\right|$ 
  of the two leading jets 
  (right) for events satisfying the EW signal selection. 
  The last bin contains all events with $\mjj > 2500\GeV$ (left) and 
  $\left|\etajj\right| > 7.5$ (right).
  The dashed line shows the expected \EWWZ contribution stacked
  on top of the backgrounds, which are shown as filled histograms. 
  The hatched bands represent the total and relative 
  statistical uncertainties on the predicted yields.
  The bottom panel shows the ratio of the number of events measured in data to the total 
  number of expected events. 
  The predicted yields are shown with their pre-fit normalizations.
          }
 \label{fig:VBSPlots}
\end{figure}

\begin{figure}[htbp]
  \centering
   \includegraphics[width=0.7\textwidth]{figures/AnalysisResults/mjj_etajj_unrolled.pdf}
    \caption{
      The one-dimensional representation of the 2D distribution of 
      $\mjj$ and $\left|\etajj\right|$, used for the EW 
      signal extraction. The x axis shows the dijet mass distribution
      in the indicated bins, split into three bins of {\etajj }: {\etajj} $\in [2.5, 4], [4, 5], \ge 5$.
      The dashed line represents the \EWWZ contribution stacked
      on top of the backgrounds, which are shown as filled histograms. 
      The hatched bands represent the total and relative 
      systematic uncertainties on the predicted yields.
      The bottom panel shows the ratio of the number of events measured in data to the total 
      number of expected events. 
      The predicted yields are shown with their best fit normalizations.
    }
  \label{fig:2DfitDistribution}
\end{figure}
\section{New physics searches}

\subsection{Limits on anomalous quartic gauge couplings}

Events satisfying the EW signal selection are used to constrain aQGCs in the effective field theory approach~\cite{Degrande:2012wf}.
Results are obtained following the formulation of Ref.~\cite{Eboli:2006wa} via
a maximum likelihood fit to the $\mt$ as described in Section~\ref{sec:aqgcProcedure}.
The $\mt$ for events satisfying the
EW signal selection is shown in Fig~\ref{fig:aQGCDistribution}. The predictions of several
indicative aQGC operators and coefficients are also shown.

The one-dimensional 95\% confidence level (CL) limits are extracted 
using the CL$\mathrm{_s}$ criterion~\cite{Junk:1999kv,CLS2,Cowan:2010js}, with all parameters
except for the coefficient being probed set to zero.
The SM prediction, including the \EWWZ process, is treated as the null hypothesis.
No deviation from the SM prediction is observed, 
and the resulting observed and expected limits are summarized in Table~\ref{tab:1Dlimits}. 

\begin{figure}[htbp]
  \centering
    \includegraphics[width=0.7\textwidth]{figures/AnalysisResults/MTWZ_aQGC.pdf}
  \caption{
      $\mt$ for events satisfying the EW signal selection,
      used to place constraints on the anomalous coupling parameters.
      The dashed lines show predictions for several aQGC parameters values that modify the \EWWZ process.
      The last bin contains all events with $\mt > 2000\GeV$.
      The hatched bands represent the total and relative 
      systematic uncertainties on the predicted yields.
      The bottom panel shows the ratio of the number of events measured in data to the total 
      number of expected events. 
      The predicted yields are shown with their best-fit normalizations from the background-only fit.
      }
 \label{fig:aQGCDistribution}
\end{figure}

\begin{table} [htbp]
\centering
\caption{Observed and expected 95\% CL limits for each operator coefficient while all other parameters are set to zero.}
\begin{tabular}{ccc}
\hline 
  Parameters & Expected limit ($\TeV^{-4}$) & Observed limit ($\TeV^{-4}$) \\ 
\hline
f$_{\text{M0}}/\Lambda^4$ & $[-11.2, 11.6]$ & $[-9.15, 9.15]$ \\  
f$_{\text{M1}}/\Lambda^4$ & $[-10.9, 11.6]$ & $[-9.15, 9.45]$ \\  
f$_{\text{S0}}/\Lambda^4$ & $[-32.5, 34.5]$ & $[-26.5, 27.5]$ \\  
f$_{\text{S1}}/\Lambda^4$ & $[-50.2, 53.2]$ & $[-41.2, 42.8]$ \\
f$_{\text{T0}}/\Lambda^4$ & $[-0.87, 0.89]$ & $[-0.75, 0.81]$ \\ 
f$_{\text{T1}}/\Lambda^4$ & $[-0.56, 0.60]$ & $[-0.49, 0.55]$ \\  
f$_{\text{T2}}/\Lambda^4$ & $[-1.78, 2.00]$ & $[-1.49, 1.85]$ \\ 
\hline 
\end{tabular} 
\label{tab:1Dlimits}
\end{table}

Constraints are also placed on aQGC parameters using a two-dimensional scan,
where two parameters are probed in the fit with all others set to zero.
The resulting 2D 95\% CL intervals for these parameters are shown in Fig.~\ref{fig:2Dlimits}.\\

\begin{figure}[htbp]
  \centering
   \includegraphics[width=0.45\textwidth]{figures/AnalysisResults/fm0_fm1_2dlimit_deltaNLL_WZ_aQGC.pdf}
   \includegraphics[width=0.45\textwidth]{figures/AnalysisResults/fs0_fs1_2dlimit_deltaNLL_WZ_aQGC.pdf}
\caption{Two-dimensional observed 95\% CL intervals (solid contour) and expected
68, 95, and 99\% CL intervals (dashed contour) on the selected aQGC parameters.
The values of coefficients 
outside of contours are excluded at the corresponding CL.
 }
 \label{fig:2Dlimits}
\end{figure}


\subsection{Limits on charged Higgs boson production}

The observed event yields by channel, and expected yields adjusted to the
best-fit values obtained from a maximum likelihood fit in the charged Higgs
boson selection are shown in Fig.~\ref{fig:higgsSignalYields}.

\begin{figure}[htbp]
  \centering
   \includegraphics[width=0.7\textwidth]{figures/AnalysisResults/yieldByChannel.pdf}
  \caption{
    Post-fit event yields in the charged Higgs boson search region.
          }
 \label{fig:higgsSignalYields}
\end{figure}

Constraints on the production of charged Higgs bosons are derived via
a maximum likelihood fit of the expected yields to the data in 
a control region of events with $\mjj > 100\GeV$ 
outside the signal selection and the $\mt$ for events the charged Higgs boson
selection. The $\mt$ distribution for the observed data and expected background,
as well as illustrated predictions for charged Higgs boson production in the 
Georgi-Machacek model are shown in Fig.~\ref{fig:higgsmt}.
No deviation from the SM prediction is observed, 
and the resulting observed and expected limits are summarized in 
Fig.~\ref{fig:higgsLimits}.
Limits are derived on the production cross section in a largely model-independent way,
assuming a narrow-width charged Higgs boson produced via vector boson fusion,
and in the Georgi-Machecek model. Results in the Georgi-Machacek model are 
presented in terms of the parameter $s_{\rm H}$ and the mass of the charged 
Higgs boson. These limits are directly comparable to other searches for charged
and doubly-charged Higgs bosons in this model.

\begin{figure}[htbp]
  \centering
    \includegraphics[width=0.7\textwidth]{figures/AnalysisResults/MTWZ_Higgs.pdf}
  \caption{
      $\mt$ for events satisfying the Higgs boson selection,
      used to place constraints on the production of charged Higgs bosons.
      The last bin contains all events with $\mt > 2000\GeV$.
      The dashed lines show predictions from the Georgi-Machechek model with
      $m(\PHpm) = 500 \,(900)\GeV$ and $s_{\PH} = 0.3 \,(0.5)$.
      The bottom panel shows the ratio of the number of events measured in data to the total 
      number of expected events. The hatched bands represent the total and relative 
      systematic uncertainties on the predicted background yields.
      The predicted yields are shown with their best-fit normalizations from the background-only fit.
      }
 \label{fig:higgsmt}
\end{figure}

\begin{figure}[htbp]
\begin{center}
  \includegraphics[width=0.45\textwidth]{figures/AnalysisResults/limits_Ind.pdf}
  \includegraphics[width=0.45\textwidth]{figures/AnalysisResults/limits_GM.pdf}
\caption{Expected (dashed lines)
  and observed (solid lines) upper limits at 95\% CL for the model independent 
  $\sigma(\mathrm{H}^{\pm}) \times \mathcal{B}(H^+\rightarrow \WZ)$ 
  as a function of $m(\mathrm{H}^\pm)$ (left) and $s_{\PH}$ in the Georgi--Machacek model (right).
  The blue shaded area covers the theoretically not allowed parameter space~\cite{Zaro:2002500}.
}
\label{fig:higgsLimits}
\end{center}
\end{figure}

