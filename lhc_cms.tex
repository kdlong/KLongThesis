\chapter{The CMS Experiment at the CERN LHC}

The Large Hadron Collider (LHC) at the European Organization for Nuclear Research (CERN)
is the world's largest and most powerful particle collider.~\cite{Evans:2008zzb} It collides protons 
at a center of mass energy of $13\TeV$, and lead ions at $2.76\TeV$ per nucleon.
Collisions of protons at the LHC are the sole focus of this thesis. 
The protons or ions are brought to collision at the center of four large detectors,
themselves a combination of many detector technologies. 
The particle detectors at the LHC were conceived, developed, and constructed over decades
by collaborations of thousands of scientists from hundreds of nations to
achieve a broad program of research. The detectors were designed and are operator 
by mutually exclusive collaborations of scientists working independently,

The detector sites are

\begin{itemize}
  \item A Large Ion Collider Experiment (ALICE) detector, located at access point 2.
  Designed for the study of lead-ion collisions, especially for the characterization
    of quark gluon plasmas formed in collisions.~\cite{Aamodt:2008zz}
  \item A Large Toroidal LHC ApparatuS (ATLAS) detector, located at access point 1.
  A general purpose detector. Designed for sensitivity to new physics
  with decays to stable SM particles, in particular the 
    discovery and characterization of the SM Higgs boson.~\cite{Aad:2008zzm}
  \item The Compact Muon Solenoid (CMS) detector, located at access point 5.
    A general purpose detector with comparable measurement potential to the ATLAS detector.~\cite{Chatrchyan:2008aa}
  \item The Large Hadron Collider Beauty (LHCb) detector, located at access point 8.
  Designed to characterize the production and decay of b-quark hadrons. Particularly
    focused on CP violation in these decays.~\cite{Alves:2008zz}
\end{itemize}

Results in this thesis are based on an analysis of proton--proton collisions at the LHC 
collected by the CMS detector in 2016. This chapter describes the design principles of the 
LHC and the CMS detector, as well as their operating characteristics in 2016.
  
\section{The Large Hadron Collider}
The LHC was proposed to the CERN council, the management group of the 
laboratory, in 1994, and accepted
with a preliminary budget in 1995. Construction was initially completed in
2008, with full operation beginning in 2009 after setbacks 
encountered in the initial commissioning.
The collider is located on the outskirts
of Geneva, Switzerland and in the nearby French countryside,
which has been the site of the CERN laboratory since it was founded in 1954.
The project was funded by the 20 member states of CERN, as well
as monetary and research contributions from many other nations participating 
in the project, including the United States.

The principle design considerations of a particle collider are the energy
of the particle collisions, the rate of collisions, and the objects collided.
The collided objects determine the possible interactions which can be probed,
while the energy of the collision drives the mass range which can be probed,
given that the produced particle mass $m$ be less than the 
center of mass energy. As particle interactions are quantum mechanical 
and fundamentally stochastic in nature, the rate of collisions is also 
critical for achieving statistically significant measurements.

Several characteristics of proton--proton collisions motivated the decision 
to use protons in the world's highest-energy collisions. Existing high-energy
accelerators, notably the Large Electron Positron Collider (LEP) collider
at CERN and the Tevatron collider at Fermilab in the United States, collided
particle--antiparticle pairs. This is advantageous for processes which proceed by annihilation, 
but requires production of positrons or antiprotons, which is a significantly
bottleneck on achieving a high rate of collisions. 

The LHC is situated in a $26.7\unit{km}$ tunnel, $45$--$170\unit{m}$
below the Swiss and French country side. The tunnel pre-dates the LHC,
having been originally built for the Large Electron Positron (LEP) collider.
It consists of 8 straight sections $528\unit{m}$ in length, and 8 arced sections.
The majority of the tunnel is $3.7\unit{m}$ in diameter, with larger excavated
areas at the four experimental caverns and other access points.
The primary motivation for an underground tunnel is to circumvent the high
cost of land acquisition, but underground operation is also advantageous for
reducing the cosmic radiation reaching the experimental cavern and  
for shielding the radiation produced by the LHC.

The protons are directed through pipes inside the tunnel, which are held
at high vacuum. The positions and accelerations of the protons are controlled 
by magnetic and electric fields maintained by instrumentation surrounding the 
vacuum pipes. The arced sections are equipped with dipole magnets,
which direct charged objects along a circular path.
The size of the LEP tunnel prohibited the installation of two independent beam
systems, which led to the adoption of a unique "two-in-one" superconducting
magnet design~\cite{}, shown in Fig~\ref{fig:dipoleXsec}. 

\begin{figure}[htbp]
  \centering
   \includegraphics[width=\textwidth]{figures/LHCandCMS/dipoleXSec.jpeg}
  \caption{
    Cross section of an LHC dipole magnet~\cite{Jean-Luc:841539}.
        }
 \label{fig:dipoleXsec}
\end{figure}

Neglecting synchrotron radiation,
the primary limitations on the energy of a synchrotron come
from the magnetic field of the bending magnets and the radius of curvature
of the tunnel. For an particle of charge $q$ with velocity $v$ accelerated at
in a circular 
trajectory of radius $R$ trough a magnetic field of strength $B$
The energy $E$ of the particle is given by

\begin{equation}
  E = eBRv \approx qBRc \,,
\label{eq:beamEnergy}
\end{equation}

where $v \approx c$, the speed of light, for $p \gg m$. Increasing the radius
requires a larger tunnel, which is limited by the cost of construction. 
The LHC tunnel size was set by the existing LEP tunnel, so achieving
a sufficient magnetic field was a major focus of the LHC development. 

Dipole magnets with a maximum field strength of $8.33\unit{T}$ were achieved
at an affordable cost through major technological advances. Such a high 
field necessitates the use of superconducting magnets. The magnets 
are constructed of niobium-titanium, and are operated at $1.9\unit{K}$,
lower than any previous accelerator. 
at 
first accelerator to operate 
In particular,
the LHC is the first accelerator to operate 
Because about 80\% of the arced sectors of the LHC are equipped with dipole magnets,
the bending radius of the dipoles is $2.804\unit{km}$, yielding the design energy
of $7\TeV$ per beam from Equation~\ref{eq:beamEnergy}.

Also brief mention of the 

How it accelerates protons (e.g., the system of accelerators)

The magnet design and characteristics 

The RF and bunch characteristics

Focusing beams to collisions


\section{Operation of the LHC in 2016}
\section{The Compact Muon Solenoid experiment}

The CMS detector is a general-purpose detector designed to 
study particle production and interactions at the TeV scale.
A major design principle of the CMS detector was the ability to 
probe the nature of electroweak symmetry breaking, which has been
achieved through the discovery and characterisation of a scalar boson
consistent with the SM Higgs boson~\cite{}. The design of the CMS detector
ensured sensitivity to a Higgs boson with mass up to
$1\TeV$ in its SM decays, as well as sensitivity to a broad class
of BSM theories. In terms of detector performance, these design principles,
as defined in Ref.~\cite{Chatrchyan:2008aa}, are:

\begin{itemize}
  \item Accurate mmuon identification and high momentum resolution, including
    dimuon mass resolution of $\approx 1\%$ at $m_{\mu\mu} = 100\GeV$ and correct charge
    assignment up to $\pt^{\mu}\approx 1\TeV$.
  \item Identification of objects with a short but significantly decay time,
    including tau leptons and b hadrons, which requires fine position resolution 
    close to the interaction point to distinguish their displaced tracks.
  \item Good electromagnetic energy and momentum resolution, achieving
    diphoton and dielectron mass resolution of $\approx 1\%$ at $m_{ee} = 100\GeV$,
    and sufficient granularity to distinguish prompt diphotons from $\pi^{0}$ decays to photons.
  \item Sufficient calorimeter resolution and hermeticity for precise dijet mass
    and missing transverse momentum reconstruction.
\end{itemize}

The CMS detector has a cylindrical geometry. As shown in Fig.~\ref{fig:CMScutaway},
it is built from the combination
of several detector technologies which work in harmony to achieve the outlined design
goals. The following sections briefly describe each of the detector subsystems
which comprise the CMS detector. 
Chapter~\ref{ch:reconstruction} details the ways in which these subsystems 
are used together to reconstruct the physics objects used for the results
presented in this thesis.

\begin{figure}[htbp]
  \centering
   \includegraphics[width=\textwidth]{figures/LHCandCMS/CMScutaway.png}
  \caption{
    A cutaway view of the CMS detector, showing each detector 
    subsystem~\cite{1742-6596-513-2-022032}.
        }
 \label{fig:CMScutaway}
\end{figure}

The coordinate system adopted by CMS, and used in this thesis,
is a right-handed coordinate system with the point $(x, y, z) = (0, 0, 0)$
centered in the detector, at the nominal collision point. The $y$-axis 
is perpendicular to the earth, where vertically upward defines the $+y$ direction.
The $x$-axis is in the plane of the LHC, with the $+x$ direction pointing towards
the center of the ring. The $z$-axis points through the center of the detector
along the direction of travel of the colliding protons, where $+z$ is defined
by the right-handedness of the coordinate system. In polar coordinates, 
the azimuthal angle $\phi$ is measured from the $x$-axis in the $x$-$y$
plane, and the polar angle $\theta$ is measured from the $z$-axis. The radial
coordinate $r$ is defined as $r = \sqrt{x^2 + y^2 + z^2}$.
Because the production of 
particles is preferentially in the forward direction (along
the $z$-axis), it is convenient to introduce the pseduorapidity $\eta = - \ln(\tan{\theta/2})$
for the polar coordinate. The momentum in the transverse direction,
defined as $\pt = \sqrt{p_{x}^2 + p_{y}^2}$, is a particularly important quantity.
because the initial transverse momentum of the collision is zero.
In this thesis, the four-momentum of an object will commonly be expressed as
$p = p(m, \vec{p}) = p(m, \pt, \eta, \phi)$.

\section{The CMS solenoidal magnet}

The central feature of the CMS apparatus is a superconducting solenoid 
of $6\unit{m}$ internal diameter and $12.5\unit{m}$ in length,
constructed from 4 layers of reinforced niobium titanium.
It provides a nearly constant $3.8\unit{T}$ magnetic field inside the solenoidal volume.
The magnetic flux of the solenoid is returned through
a $12 000$-tonne steel yoke, comprising 5 wheels and 2 endcaps,
which is fully saturated to approximately $2\unit{T}$. 
The tracker and calorimeters are situated inside the solenoid, while 
the muon detectors are embedded in the steel return yoke, as shown
in Fig.~\ref{fig:CMScutaway}.

During operation, the CMS solenoid stores approximately $2.4\unit{GJ}$ of 
energy, the largest magnet in the world by this metric. This immense
size leads to a powerful bending radius for charged objects within the detector,
which is a critical metric for particle reconstruction and identification.
In particular, objects of charge $q$ moving
in a magnetic field $\vec{B}$ at velocity $\vec{v}$ experience a Lorentz force,

\begin{equation}
  \vec{F} = q\vec{B} \times \vec{v} \,.
\end{equation}

Because the solenoidal field is constant and aligned along the $z$-direction, 
$\vec{B} = B\hat{z}$, the force is purely in the transverse direction.
Consequently, charged particles within the CMS
solenoid travel along a nearly helical path of radius $R=\pt/\abs{q}B$, where 
small deviations from this path arise from non-uniformity of the field
and interactions in the detector material. The sign of a 
$\pt$ of a particle of known charge can therefore be deduced by measuring its radius of
curvature. The bending power of the magnet $BR$ is therefore a critical 
parameter determining the detector capability for accurate particle $\pt$ measurement.
The nearly $12\unit{Tm}$ bending power achieved by the CMS is critical to achieving
the high momentum resolution outlined in the design goals of the experiment. It
is also fundamental to the particle flow reconstruction technique utilized by CMS,
outlined in Chapter~\ref{ch:reconstruction}.

\section{The CMS silicon tracking system}
\section{The CMS electromagnetic calorimeter}
\section{The CMS hadronic calorimeter}
\section{The CMS muon system}
\subsection{Drift tube system}
\subsection{Resistive plate chamber system}
\subsection{Cathode strip chamber system}

