\chapter{Analysis Strategy}
A broad introduction
\section{WZ VBS event signature and background composition}
Give a cut flow, and the expected background composition. Show Feynman diagrams.
\section{Event triggering}
Describe the triggers, the trigger efficiency, and the prefire efficiency. Make some plots!
\section{Event selection}

Quick overview of the selections. Do you really need a separate section for each? I think not

\begin{table}[!ht]
  \begin{center}
  \caption{Summary of event selections and fiducial region definitions for the analysis. 
    The selections labeled ``EW signal'' and ``Higgs boson'' are applied to data and reconstructed 
    simulated events.
    The EW signal selection is used for all measurements except for charged the charged Higgs boson search,
    which uses the selection indicated in the column labeled ``Higgs boson''.
    The \WZjj cross section is reported in the fiducial regions defined by the selections specified in 
    the last two columns, which are applied to simulated events.
    $n_{\jet}$, $n_{\mathrm{b}}$, and $\pt^{\mathrm{b}}$ refer to the number of
    anti-\kt jets, the number of anti-\kt b-tagged jets, and the b jet 
    \pt threshold, respectively. Other variables are defined in the text.
    }
  \begin{tabular}{c|c|c|c|c}
  \hline
                                    & EW signal & Higgs boson & Tight fiducial & Loose fiducial\\
    \hline\hline
    $  \PT^{\ell'_{1}}   $ [GeV]    & $> 25$    & $> 25$        & $ > 25 $       & $ > 20 $ \\
    $  \PT^{\ell'_{2}}   $ [GeV]    & $> 15$    & $> 15$        & $ > 15 $       & $ > 20 $ \\
    $  \PT^{\ell}     $ [GeV]       & $> 20$    & $> 20$        & $ > 20 $       & $ > 20 $ \\
  $\left|\eta^{\mu}\right|   $      & $< 2.4$   & $< 2.4$       & $ < 2.5$       & $ < 2.5$ \\
    $\left|\eta^{\mathrm{e}}\right|$      & $< 2.5$   & $< 2.5$       & $ < 2.5$       & $ < 2.5$ \\
  $\left|m_{\ell'\ell'}-m_{\Z}\right|$ [GeV] & $ < 15 $ & $ < 15 $ & $ < 15 $ & $ < 15 $ \\
  $m_{3\ell}                $ [GeV] & $> 100$   & $> 100$       & $> 100$        & $> 100$    \\
  $m_{\ell\ell}           $ [GeV]   & $> 4$     & $> 4$         & $>4$           & $>4$    \\
  $\ptmiss                  $ [GeV] & $> 30$       & $> 30$     &   -            &   -     \\
  $\left|\eta^{\jet}\right|  $      & $< 4.7$   & $< 4.7$       & $< 4.7$        & $< 4.7$ \\
    $\PT^{\jet}                $ [GeV] & $ > 50$   & $> 30$        & $> 50$         & $> 30$  \\
  $\left|\Delta R(\jet, \ell)\right|$            & $ > 0.4$  & $> 0.4$       & $> 0.4$        & $> 0.4$ \\
  $n_{\mathrm{\jet}}           $    & $\ge 2$   & $\ge 2$       & $\ge 2$        & $\ge 2$    \\
  $\PT^{\mathrm{b}}         $ [GeV] & $ > 30$   & $ > 30$       &   -            &   -     \\
  $n_{\mathrm{b}}       $         & $= 0$     & $= 0$         &   -            &   -     \\
  $\mjj             $               & $> 500$   & $> 500$       & $> 500$        & $> 500$ \\
  $\left|\etajj \right|$            &$> 2.5$         & $> 2.5$ & $> 2.5$ & $> 2.5$ \\
  $\left| \zepl \right|$            & $< 2.5$ & - & $< 2.5$ & - \\
  \end{tabular}
  \label{tab:selections}
  \end{center}
\end{table}

\section{Background contributions}
Background contributions in this analysis are divided into two categories:
background processes with prompt isolated leptons, \eg, 
$\PZ\PZ$, \tZq, $\ttbar\cPZ$;
and background processes with nonprompt leptons from hadrons decaying to leptons inside jets or
jets misidentified as isolated leptons, primarily $\ttbar$ and $\cPZ$+jets.
The background processes with prompt leptons are estimated from MC simulation, while backgrounds
with nonprompt leptons from hadronic activity are estimated from control samples in data.
The nonprompt component of the $\cPZ\gamma$ process,
in which the photon experiences conversion into leptons in the tracker,
is evaluated using MC simulation. 
\subsection{Simulation and validation of signal and prompt backgrounds}

Several Monte Carlo (MC) event generators are used to simulate the signal and
background processes.

The EW-induced production of \WZ boson pairs and two final-state quarks, where the W and Z bosons decay leptonically, 
is simulated at leading order (LO) in perturbative QCD using 
\MG v2.4.2~\cite{MGatNLO}. 
The sample includes all contributions to the four-lepton final state at $\mathcal{O}(\alpha^6)$ and 
with a resonant W boson propagator, including
triboson processes, where the \WZ boson pair is accompanied by 
a third vector boson that decays into jets, as well as diagrams involving the quartic coupling vertex. 
The resonant W boson is decayed using {\sc MadSpin}~\cite{Artoisenet:2012st}.
Contributions with an initial-state b quark are excluded from the sample
as they are considered part of the \tZq process.
The predictions from this sample are cross-checked with LO predictions from the 
event generators \VBFNLO~3.0~\cite{VBFNLO} and 
\Sherpa~v2.2.4~\cite{Gleisberg:2008ta,Gleisberg:2003xi}, 
and with fixed-order calculations from
\Moca~\cite{leshouches2017,Recola}. Agreement is obtained when using equivalent configurations
of input parameters, including couplings, particle masses and widths, and the choice of
renormalization ($\mu_{R}$) and 
factorization scales ($\mu_{F}$),
and differences arising from such configurations are considered when assessing the uncertainty.

Several MC simulations of the \QCDWZ process are considered.
The simulations are inclusive in the number of jets associated with the 
leptonically decaying \W and \Z bosons, and therefore include 
the \WZjj contribution studied in this analysis.
The primary MC sample is simulated at 
LO with \MG v2.4.2, with contributions to \WZ production with up to three outgoing partons 
included in the matrix element calculation. 
The different jet multiplicities are merged using the MLM scheme~\cite{MLMmerging}.
A next-to-LO (NLO) sample from \MG v2.3.3 
with zero or one outgoing partons at Born level, merged using the \FxFx scheme~\cite{Frederix:2012ps},
and an inclusive NLO sample from \POWHEG2.0~\cite{Melia:2011tj,Nason:2004rx,Frixione:2007vw,powheg:2010}
are also considered. 
The LO MC with MLM merging, referred to as the MLM-merged sample, 
is taken as the central prediction due its inclusion of
\WZ plus three-parton contributions at tree level, which are relevant
to \WZjj production.
The other samples,
which are used to access the modeling uncertainty in the \QCDWZ process,
are referred to as the \FxFx-merged
and the \POWHEG samples, respectively.
Each sample is scaled to the inclusive NLO cross section from \POWHEG2.0.

In addition to the \EWWZ and \QCDWZ process, which at tree level are 
$\mathcal{O}(\alpha^4)$ and $\mathcal{O}(\alpha^2\alpha_{S}^2)$ respectively,
a smaller contribution at $\mathcal{O}(\alpha^3\alpha_{S})$ 
contributes to the \WZjj state. We refer to this contribution as the 
interference term. It is evaluated using samples of particle-level
events generated with \MG v2.6.0. Samples are generated with the dynamic $\mu_{R}$
and $\mu_{F}$ set to the maximum of the parton transverse momenta per event, and with fixed
scales $\mu_{R} = \mu_{F} = m_{\mathrm{W}}$, where $m_{\mathrm{W}}$ is the world average value of the 
W boson mass, taken from Ref.~\cite{Tanabashi:2018oca}.

The associated production of a $\PZ$ boson and a single top quark, referred to as \tZq production,
is simulated at NLO in the four-flavor scheme using \MG v2.3.3. 
The sample is scaled using a cross section computed at NLO with \MG in the five-flavor scheme, 
following the procedure of Ref.~\cite{Sirunyan:2017nbr}.

The production of $\PZ$ boson pairs via $\Pq\Paq$ annihilation is generated at NLO in perturbative QCD with
\POWHEG2.0 while the $\Pg\Pg \to \ZZ$ process is simulated at LO with \MCFM7.0~\cite{Campbell:2011bn}.
The $\ZZ$ samples are scaled to the cross section calculated at next-to-NLO
for $\Pq\Paq \to \ZZ$ \cite{Cascioli:2014yka} ($K$ factor 1.1)
and at NLO for $\Pg\Pg \to \ZZ$ \cite{Caola:2015psa} ($K$ factor 1.7).
The $\cPZ\Pgg$, $\ttbar\text{V}$ ($\ttbar\PW$, $\ttbar\PZ$), $\PQt\PZ$, 
and triboson events VVV (WWZ, WZZ, ZZZ)  
are generated at NLO with \MG v2.3.3, with the vector bosons generated on-shell
and decayed via {\sc MadSpin}.

The simulation of the aQGC processes is performed at LO using \MG v2.4.2 and employs matrix element 
reweighting to obtain a finely spaced grid of parameters for each of the anomalous couplings
operators probed by the analysis. The configuration of input parameters is equivalent to that used for the 
\EWWZ sample described previously. 
The production of charged Higgs bosons in the Georgi-Machacek model
is simulated using \MG~v2.3.3.

The \PYTHIA~v8.212~\cite{Sjostrand:2006za,Sjostrand:2015} package
is used for parton showering, hadronization, and
the underlying event simulation, with parameters set by the CUETP8M1
tune~\cite{Khachatryan:2015pea} for all simulated samples.
For the \EWWZ process, comparisons are made at particle-level with the parton shower
and hadronization 
of \Sherpa and with \Herwig~v7.1~\cite{Bellm:2015jjp,Bahr:2008pv}.
For all MC simulations used in this analysis, the NNPDF3.0~\cite{NNPDF2015} set of
parton distribution functions (PDFs) is used, with PDFs calculated to the same
order in perturbative QCD as the hard scattering process. 

The detector response is simulated using a detailed
description of the CMS detector implemented in the \GEANTfour
package~\cite{GEANT, Geant2}. The simulated events are  reconstructed
using the same algorithms as used for the data. 
The simulated samples include additional interactions in the same and neighboring bunch crossings,
referred to as pileup.
Simulated events are weighted so that the pileup distribution reproduces that observed in 
the data, which has an average of about 23 interactions per bunch
crossing.

\subsection{Estimation of nonprompt backgrounds}
The contribution from background processes with nonprompt leptons is
evaluated using control regions of events in data using the 
technique described in 
Refs.~\cite{Khachatryan:2016tgp,Sirunyan:2017sbn}.
Events satisfying the full analysis selection,
with the exception that one, two, or three leptons pass relaxed identification
requirements but fail the more stringent requirements applied to signal events,
are selected to form seven control regions. The control regions are
mutually independent and additionally independent from the signal selection.
The small contribution to the control regions from events with three prompt leptons
is estimated with MC simulation and 
subtracted from the event samples.

The expected contribution in the signal region is estimated
using ``loose-to-tight'' efficiency factors
applied to the lepton candidates failing the analysis requirements 
in the control region events.
The efficiency factors are calculated from a
sample of $\PZ$+$\ell_{\mathrm{candidate}}$ events, where \PZ
denotes a pair of oppositely charged, same-flavor leptons satisfying the
full identification requirements and $\left|m_{\ell^{+}\ell^{-}} - m_\PZ\right| < 10\GeV$.
The loose-to-tight efficiency factors 
are obtained from ratios of events where the $\ell_{\mathrm{candidate}}$ 
object satisfies the full identification requirements 
to events where all identification criteria are not satisfied, and
is parameterized as a function of \PT and $\eta$. 
A cross-check of the technique is performed by 
repeating the procedure with efficiency factors derived from a 
sample of events dominated by dijet production. The 
loose-to-tight efficiency factors obtained in the two regions
agree to within 30\% for the full \PT and $\eta$ range.

This method is validated in nonoverlapping data samples enriched in Drell--Yan and $\ttbar$ contributions.
The Drell--Yan region is defined by inverting the selection requirement in $\ptmiss$, and
the $\ttbar$ region is defined by requiring at least one b-tagged jet and rejecting events with $|m_{\ell'\ell'} - m_\PZ| < 5\GeV$
while keeping all other requirements for the signal region.
The overall yield predicted agrees with that 
measured in the control regions to within 20\%.

The event count of the control samples of nonprompt events
satisfying the EW signal selection limits  
differential predictions in this region. 
The combined shape of the nonprompt background for all channels 
is therefore used for each channel in the \EWWZ signal strength
measurement and in the extraction of constraints on anomalous quartic
gauge couplings.
The normalization of the distribution per channel is taken from the 
ratio of the nonprompt yield in a single channel to the total nonprompt event yield 
measured in \WZjj events with no requirements on the dijet system.
These ratios are found to be consistent within the statistical uncertainty with ratios measured
when relaxing the jet \PT requirement in \WZjj events, in \WZ events inclusive in the number of jets, 
and in events satisfying the EW signal and control region selections. 

\section{Statistical procedures}
Overview of the likelihood and assymptotic formulae
\subsection{WZjj cross section measurement}
Briefly describe likelihood
\subsection{Significance of electroweak WZ Production}
Briefly describe 2D fit
\subsection{Limits on charged Higgs bosons}
Briefly describe fit to MT
\subsection{Limits on anomalous quartic gauge couplings}

\section{Systematic uncertainties}

\begin{table}[htbp]
     \centering
     \caption{ The dominant uncertainty contributions in the fiducial 
         \WZjj cross section measurement 
         and their expected contributions to the significance of the
         \EWWZ signal strength measurement. The impact of each systematic 
         uncertainty in the \WZjj 
         cross section measurement is obtained by freezing the set of associated nuisance 
         parameters to their best fit values and comparing the total uncertainty in the signal strength
         to the result from the nominal fit. 
         The effect on the \EWWZ significance, shown in the last column,
         is defined as the relative increase in the expected significance when
         freezing the nuisance term to its best fit value.
           }
     \begin{tabular}{l|ccc}
 \hline %------------------------------------------------------------------------------------------
     Source of systematic uncertainty & \multicolumn{3}{c}{Relative systematic uncertainty [\%]} \\
                                      & $\sigma_{\WZjj}$ & \EWWZ significance \\
 \hline %------------------------------------------------------------------------------------------
 \hline %------------------------------------------------------------------------------------------
 Jet energy scale                     & $+10.7\, /-8.1$ & $ 7.0 $               \\ %done
 Jet energy resolution                & $+1.9\,/-2.1$   & $< 0.1$             \\ %done
 \QCDWZ modeling                      &    N/A          & $ 2.2 $             \\
 Other background theory              &  $+2.2\,/-2.2$  & $ 0.3 $             \\ %done
 Nonprompt normalization              &  $+2.5\,/-2.5$  & $ 0.3 $             \\ %done
 Nonprompt event count                &  $+6.0\,/-5.8$  & $ 1.7 $               \\ %done
 Lepton energy scale and eff.         &  $+3.5\,/-2.7$  & $< 0.1$             \\ %done
 b tagging                            &  $+2.0\,/-1.7$  & $< 0.1$             \\ %done
 Integrated luminosity                &  $+3.6\,/-3.0$  & $< 0.1$             \\ %done
 \hline %------------------------------------------------------------------------------------------
      \end{tabular}
     \label{tab:systematics}
\end{table}

