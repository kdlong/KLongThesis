\chapter{Analysis Strategy}
A broad introduction
\section{WZ VBS event signature and background composition}
Give a cut flow, and the expected background composition. Show Feynman diagrams.
\section{Event triggering}
Describe the triggers, the trigger efficiency, and the prefire efficiency. Make some plots!
\section{Event selection}

Quick overview of the selections. Do you really need a separate section for each? I think not

\begin{table}[!ht]
  \begin{center}
  \caption{Summary of event selections and fiducial region definitions for the analysis. 
    The selections labeled ``EW signal'' and ``Higgs boson'' are applied to data and reconstructed 
    simulated events.
    The EW signal selection is used for all measurements except for charged the charged Higgs boson search,
    which uses the selection indicated in the column labeled ``Higgs boson''.
    The \WZjj cross section is reported in the fiducial regions defined by the selections specified in 
    the last two columns, which are applied to simulated events.
    $n_{\jet}$, $n_{\mathrm{b}}$, and $\pt^{\mathrm{b}}$ refer to the number of
    anti-\kt jets, the number of anti-\kt b-tagged jets, and the b jet 
    \pt threshold, respectively. Other variables are defined in the text.
    }
  \begin{tabular}{c|c|c|c|c}
  \hline
                                    & EW signal & Higgs boson & Tight fiducial & Loose fiducial\\
    \hline\hline
    $  \PT^{\ell'_{1}}   $ [GeV]    & $> 25$    & $> 25$        & $ > 25 $       & $ > 20 $ \\
    $  \PT^{\ell'_{2}}   $ [GeV]    & $> 15$    & $> 15$        & $ > 15 $       & $ > 20 $ \\
    $  \PT^{\ell}     $ [GeV]       & $> 20$    & $> 20$        & $ > 20 $       & $ > 20 $ \\
  $\left|\eta^{\mu}\right|   $      & $< 2.4$   & $< 2.4$       & $ < 2.5$       & $ < 2.5$ \\
    $\left|\eta^{\mathrm{e}}\right|$      & $< 2.5$   & $< 2.5$       & $ < 2.5$       & $ < 2.5$ \\
  $\left|m_{\ell'\ell'}-m_{\Z}\right|$ [GeV] & $ < 15 $ & $ < 15 $ & $ < 15 $ & $ < 15 $ \\
  $m_{3\ell}                $ [GeV] & $> 100$   & $> 100$       & $> 100$        & $> 100$    \\
  $m_{\ell\ell}           $ [GeV]   & $> 4$     & $> 4$         & $>4$           & $>4$    \\
  $\ptmiss                  $ [GeV] & $> 30$       & $> 30$     &   -            &   -     \\
  $\left|\eta^{\jet}\right|  $      & $< 4.7$   & $< 4.7$       & $< 4.7$        & $< 4.7$ \\
    $\PT^{\jet}                $ [GeV] & $ > 50$   & $> 30$        & $> 50$         & $> 30$  \\
  $\left|\Delta R(\jet, \ell)\right|$            & $ > 0.4$  & $> 0.4$       & $> 0.4$        & $> 0.4$ \\
  $n_{\mathrm{\jet}}           $    & $\ge 2$   & $\ge 2$       & $\ge 2$        & $\ge 2$    \\
  $\PT^{\mathrm{b}}         $ [GeV] & $ > 30$   & $ > 30$       &   -            &   -     \\
  $n_{\mathrm{b}}       $         & $= 0$     & $= 0$         &   -            &   -     \\
  $\mjj             $               & $> 500$   & $> 500$       & $> 500$        & $> 500$ \\
  $\left|\etajj \right|$            &$> 2.5$         & $> 2.5$ & $> 2.5$ & $> 2.5$ \\
  $\left| \zepl \right|$            & $< 2.5$ & - & $< 2.5$ & - \\
  \end{tabular}
  \label{tab:selections}
  \end{center}
\end{table}

\section{Background contributions}
Background contributions in this analysis are divided into two categories:
background processes with prompt isolated leptons, \eg, 
$\PZ\PZ$, \tZq, $\ttbar\cPZ$;
and background processes with nonprompt leptons from hadrons decaying to leptons inside jets or
jets misidentified as isolated leptons, primarily $\ttbar$ and $\cPZ$+jets.
The background processes with prompt leptons are estimated from MC simulation, while backgrounds
with nonprompt leptons from hadronic activity are estimated from control samples in data.
The nonprompt component of the $\cPZ\gamma$ process,
in which the photon experiences conversion into leptons in the tracker,
is evaluated using MC simulation. 
\subsection{Simulation and validation of signal and prompt backgrounds}

Several Monte Carlo (MC) event generators are used to simulate the signal and
background processes.

The EW-induced production of \WZ boson pairs and two final-state quarks, where the W and Z bosons decay leptonically, 
is simulated at leading order (LO) in perturbative QCD using 
\MG v2.4.2~\cite{MGatNLO}. 
The sample includes all contributions to the four-lepton final state at $\mathcal{O}(\alpha^6)$ and 
with a resonant W boson propagator, including
triboson processes, where the \WZ boson pair is accompanied by 
a third vector boson that decays into jets, as well as diagrams involving the quartic coupling vertex. 
The resonant W boson is decayed using {\sc MadSpin}~\cite{Artoisenet:2012st}.
Contributions with an initial-state b quark are excluded from the sample
as they are considered part of the \tZq process.
The predictions from this sample are cross-checked with LO predictions from the 
event generators \VBFNLO~3.0~\cite{VBFNLO} and 
\Sherpa~v2.2.4~\cite{Gleisberg:2008ta,Gleisberg:2003xi}, 
and with fixed-order calculations from
\Moca~\cite{leshouches2017,Recola}. Agreement is obtained when using equivalent configurations
of input parameters, including couplings, particle masses and widths, and the choice of
renormalization ($\mu_{R}$) and 
factorization scales ($\mu_{F}$),
and differences arising from such configurations are considered when assessing the uncertainty.

Several MC simulations of the \QCDWZ process are considered.
The simulations are inclusive in the number of jets associated with the 
leptonically decaying \PW and \cPZ bosons, and therefore include 
the \WZjj contribution studied in this analysis.
The primary MC sample is simulated at 
LO with \MG v2.4.2, with contributions to \WZ production with up to three outgoing partons 
included in the matrix element calculation. 
The different jet multiplicities are merged using the MLM scheme~\cite{MLMmerging}.
A next-to-LO (NLO) sample from \MG v2.3.3 
with zero or one outgoing partons at Born level, merged using the \FxFx scheme~\cite{Frederix:2012ps},
and an inclusive NLO sample from \POWHEG2.0~\cite{Melia:2011tj,Nason:2004rx,Frixione:2007vw,powheg:2010}
are also considered. 
The LO MC with MLM merging, referred to as the MLM-merged sample, 
is taken as the central prediction due its inclusion of
\WZ plus three-parton contributions at tree level, which are relevant
to \WZjj production.
The other samples,
which are used to access the modeling uncertainty in the \QCDWZ process,
are referred to as the \FxFx-merged
and the \POWHEG samples, respectively.
Each sample is scaled to the inclusive NLO cross section from \POWHEG2.0.

In addition to the \EWWZ and \QCDWZ process, which at tree level are 
$\mathcal{O}(\alpha^4)$ and $\mathcal{O}(\alpha^2\alpha_{S}^2)$ respectively,
a smaller contribution at $\mathcal{O}(\alpha^3\alpha_{S})$ 
contributes to the \WZjj state. We refer to this contribution as the 
interference term. It is evaluated using samples of particle-level
events generated with \MG v2.6.0. Samples are generated with the dynamic $\mu_{R}$
and $\mu_{F}$ set to the maximum of the parton transverse momenta per event, and with fixed
scales $\mu_{R} = \mu_{F} = m_{\mathrm{W}}$, where $m_{\mathrm{W}}$ is the world average value of the 
W boson mass, taken from Ref.~\cite{Tanabashi:2018oca}.

The associated production of a $\PZ$ boson and a single top quark, referred to as \tZq production,
is simulated at NLO in the four-flavor scheme using \MG v2.3.3. 
The sample is scaled using a cross section computed at NLO with \MG in the five-flavor scheme, 
following the procedure of Ref.~\cite{Sirunyan:2017nbr}.

The production of $\PZ$ boson pairs via $\cPq\cPaq$ annihilation is generated at NLO in perturbative QCD with
\POWHEG2.0 while the $\Pg\Pg \to \ZZ$ process is simulated at LO with \MCFM7.0~\cite{Campbell:2011bn}.
The $\ZZ$ samples are scaled to the cross section calculated at next-to-NLO
for $\cPq\cPaq \to \ZZ$ \cite{Cascioli:2014yka} ($K$ factor 1.1)
and at NLO for $\Pg\Pg \to \ZZ$ \cite{Caola:2015psa} ($K$ factor 1.7).
The $\cPZ\cPgg$, $\ttbar\text{V}$ ($\ttbar\PW$, $\ttbar\PZ$), $\PQt\PZ$, 
and triboson events VVV (WWZ, WZZ, ZZZ)  
are generated at NLO with \MG v2.3.3, with the vector bosons generated on-shell
and decayed via {\sc MadSpin}.

The simulation of the aQGC processes is performed at LO using \MG v2.4.2 and employs matrix element 
reweighting to obtain a finely spaced grid of parameters for each of the anomalous couplings
operators probed by the analysis. The configuration of input parameters is equivalent to that used for the 
\EWWZ sample described previously. 
The production of charged Higgs bosons in the Georgi-Machacek model
is simulated using \MG~v2.3.3.

The \PYTHIA~v8.212~\cite{Sjostrand:2006za,Sjostrand:2015} package
is used for parton showering, hadronization, and
the underlying event simulation, with parameters set by the CUETP8M1
tune~\cite{Khachatryan:2015pea} for all simulated samples.
For the \EWWZ process, comparisons are made at particle-level with the parton shower
and hadronization 
of \Sherpa and with \Herwig~v7.1~\cite{Bellm:2015jjp,Bahr:2008pv}.
For all MC simulations used in this analysis, the NNPDF3.0~\cite{NNPDF2015} set of
parton distribution functions (PDFs) is used, with PDFs calculated to the same
order in perturbative QCD as the hard scattering process. 

The detector response is simulated using a detailed
description of the CMS detector implemented in the \GEANTfour
package~\cite{GEANT, Geant2}. The simulated events are  reconstructed
using the same algorithms as used for the data. 
The simulated samples include additional interactions in the same and neighboring bunch crossings,
referred to as pileup.
Simulated events are weighted so that the pileup distribution reproduces that observed in 
the data, which has an average of about 23 interactions per bunch
crossing.

\subsection{Estimation of nonprompt backgrounds}

The contribution from background processes with nonprompt leptons is
evaluated with control regions of events in data.
Events satisfying the full analysis selection,
with the exception that one, two, or three leptons pass relaxed identification
requirements but fail the more stringent requirements applied to signal events,
are selected to form seven control regions. The control regions are
mutually independent and additionally independent from the signal selection.
The small contribution to the control regions from events with three prompt leptons
is estimated with MC simulation and 
subtracted from the event samples.

\begin{figure}[htbp]
  \centering
   \includegraphics[width=0.45\textwidth]{figures/AnalysisProcedure/mjj_PPF.pdf}
   \includegraphics[width=0.45\textwidth]{figures/AnalysisProcedure/mjj_PFP.pdf}
  \caption{
    The PPF (left) and PFP (right) control regions used in the nonprompt estimation
    for events satisfying the \WZjj selection.
    The nonprompt MC simulations are shown for reference, and demonstrate that the
    PPF region is dominated by Drell--Yan events while the PFP region
    is dominated by {\ttbar} events.
        }
 \label{fig:nonpromptBackgroundCRs}
\end{figure}

The control regions are defined by the identification
criteria of the three selected lepton candidates. 
A lepton candidate is labeled as ``passed'' (P) if it passes the ``tight'' identification
requirements required for signal events, 
and ``failed'' (F) if it passes the relaxed lepton identification criteria, 
but fails the tight identification. The signal region is defined by events 
with three passing leptons, referred to as PPP events. The PPF region, where the first
two labels refer to the two leptons which satisfy the tight identification criteria and are
associated with the \cPZ boson candidate, and the failing lepton is associated with
the \cPW-candidate, is dominated by {\Zpj} events.
The $\mjj$ for the observed data compared to the prediction of the MC simulation
in the \WZjj selection, with $\pt^{\jet} > 30\GeV$, in two illustrative control regions are shown
in Fig.~\ref{fig:nonpromptBackgroundCRs}. The nonprompt estimation does not depend on the 
nonprompt MC simulations, including Drell--Yan and $\ttbar$ production, which are shown for reference.
All possible combinations of P and F objects are listed in Table~\ref{tab:control_regions},
along with the dominant production mechanisms of such events.

\begin{table}[htbp]
    \centering
    \caption{Possible combinations of lepton identification for three-lepton events
            defining the control regions used to estimate nonprompt backgrounds.}
    \begin{tabular}{c|ccc} 
\hline %------------------------------------------------------------------------------------------
Dominant        & \multicolumn{3}{c}{Combination}                                                  \\
contribution    & \multicolumn{2}{c}{Z-candidate leptons}   & W-candidate lepton                   \\
\hline %------------------------------------------------------------------------------------------
\hline %------------------------------------------------------------------------------------------
QCD multiet    & F & F & F \\ 
\hline %----------------------------------------------------------------------------------------- 
W+jets         & F & F & P \\
               & P & F & F \\
               & F & P & F \\
\hline %----------------------------------------------------------------------------------------- 
Z+jets         & P & P & F \\
\hline %----------------------------------------------------------------------------------------- 
\ttbar, other  & P & F & P \\
               & F & P & P \\
     \end{tabular}
    \label{tab:control_regions}
\end{table}

The expected contribution from these processes in the signal region is estimated
using ``loose-to-tight'' efficiency factors, which are
applied as transfer factors to the lepton candidates failing the analysis requirements 
in the control region events.
The efficiency factors, referred to as ``fake rates,'' are 
evaluated using samples of events dominated by jet activity.
A sample of $\PZ$+$\lcand$ events is defined by 
selecting events with exactly three leptons satisfying
the loose identification criteria, using the same trigger requirements as
for signal events.
Two opposite sign, same flavor 
leptons satisfying the tight identification and consistent with the decay of
a \PZ boson, specifically $\left|m_{\ell^{+}\ell^{-}} - m_\PZ\right| < 10\GeV$,
where $m_{\PZ}$ denotes the nominal \PZ boson mass, are required. To reduce contamination
from \WZ events, the events must additionally have $\ptmiss < 25\GeV$
and $m_{\rm T}(\ell_{3}, \ptmiss) < 25\GeV$. The lepton not associated with the 
\PZ boson candidate is taken as the {\lcand} object. 

\begin{figure}[htbp]
  \centering
   \includegraphics[width=0.4\textwidth]{figures/AnalysisProcedure/ratio1DPt_allE.pdf}
   \includegraphics[width=0.4\textwidth]{figures/AnalysisProcedure/ratio1DPt_allMu.pdf}
   \includegraphics[width=0.4\textwidth]{figures/AnalysisProcedure/ratio1DPt_allE_MC.pdf}
   \includegraphics[width=0.4\textwidth]{figures/AnalysisProcedure/ratio1DPt_allMu_MC.pdf}
  \caption{
    Ratio of events in which the $\ell_{\rm candidate}$ objects passes the tight lepton 
    identification criteria to candidates passing only the loose lepton identification 
    criteria in bins of the {\lcand} \PT for the selection of $\PZ$+$\ell_{\mathrm{candidate}}$ 
    events described in the text.
    The events in which the $\ell_{\rm candidate}$ object is a electron (muon) are shown in 
    the left (right) plot. The top plots show events selected in data, with
    corrections for contamination of true prompt events from MC simulation. The bottom plots
    show predictions from the {\Zpj} and $\ttbar$ MC samples.
          }
 \label{fig:fakeRates1DPt}
\end{figure}

The fake rates are evaluated 
using ratios of events where the $\ell_{\mathrm{candidate}}$ 
object satisfies the full identification requirements 
to events where all identification criteria are not satisfied, defined as
$\varepsilon_{\text{fake}} = \frac{N_{\text{tight}}}{N_\text{loose}}$,
is parameterized as a function of the {\lcand} \PT and $\eta$. 
The expected contribution from true three-prompt-lepton events is
estimated from MC simulation and subtracted from the event samples.
The fake rates measured in data, corrected using the prompt MC simulation,
and the predicted fake rates from the nonprompt MC simulation are shown in
Fig.~\ref{fig:fakeRates1DPt}, which are seen to be in good agreement.
The MC simulation prediction is not used in the analysis, but serves as a consistency-check
of the nonprompt estimation technique.
The two-dimensional values of the fake rates derived from data which are used for
the analysis are shown in Fig.~\ref{fig:fakeRates2D}.

\begin{figure}[htbp]
  \centering
   \includegraphics[width=0.45\textwidth]{figures/AnalysisProcedure/ratio2D_allE.pdf}
   \includegraphics[width=0.45\textwidth]{figures/AnalysisProcedure/ratio2D_allMu.pdf}
  \caption{
    Ratio of events in data, with corrections for contamination of true prompt events from MC simulation,
    in which the $\ell_{\rm candidate}$ objects passes the tight lepton 
    identification criteria to candidates passing only the loose lepton identification 
    criteria in bins of \PT and $\eta$ for the selection of $\PZ$+$\ell_{\mathrm{candidate}}$ 
    events described in the text.
    The events in which the $\ell_{\rm candidate}$ object is a electron (muon) are shown in 
    the left and right plots respectively. 
        }
 \label{fig:fakeRates2D}
\end{figure}


A cross-check of the technique is performed by 
repeating the procedure with efficiency factors derived from a 
sample of events dominated by dijet production. 
Events passing single lepton triggers are selected, requiring at least one jet and one
{\lcand} object passing the loose identification. Events with additional loose leptons are not selected.
The loose lepton must be separated from the jet with highest \pt by $\Delta R(\lcand, \jet) > 1$.
The contamination of WZ is suppressed by requiring $E_T^{\text{miss}} < 25\,\GeV$  and  $m_T < 25\,\GeV$. 
The loose-to-tight efficiency factors obtained in the two regions
agree to within 30\% for the full \PT and $\eta$ range.

\begin{table}[htbp]
    \centering
    \begin{tabular}{|c|c|c|c|c|c|c|c| }
\hline %------------------------------------------------------------------------------------------ 
Control region &  \multicolumn{7}{c|}{Includes regions}                                                                 \\
\hline %------------------------------------------------------------------------------------------ 
FPP & FPP &     &     & FFF & FPF & FFP &        \\
PFP &     &     & PFP & FFF &     & FFP & PFF    \\
PPF &     & PPF &     & FFF & FPF &     & PFF    \\
FPF &     &     &     & FFF & FPF &     &        \\
FFP &     &     &     & FFF &     & FFP &        \\
PFF &     &     &     & FFF &     &     & PFF    \\
FFF &     &     &     & FFF &     &     &        \\
\hline %------------------------------------------------------------------------------------------ 
     \end{tabular}
    \caption{ Possible nonprompt control regions and regions of overlap.}
    \label{tab:nonpromptRegionOverlap}
\end{table}

Because control regions with $n_{F}$ failing leptons can contribute to the region with
$n_{F}-1$ failing leptons,
the total background contribution in the signal region is not a simple sum of the 
control region event yields multiplied by the fake rate of each failing leptons. 
Table~\ref{tab:nonpromptRegionOverlap} shows the possible contributions 
of control regions with loose leptons into tight lepton regions.
For example,
the estimated contribution of events to the signal region $N_{B,FPP+PFP}$ 
from the FPP and PFP control region, with $N_{FPP}$ and $N_{PFP}$ data events is given by

\begin{multline*}
N_{B,FPP + PFP} = N_{FPP} \times f_{l1}
  - \underbrace{N_{FFF} \times f_{11} \times f_{l2} \times f_{l3}}_{\textrm{interference of FPP and FFF}} \\
  - \underbrace{f_{l1} \times f_{l3} \times (N_{FPF} - N_{FFF} \times f_{l2} )}_{\textrm{interference of FPP and FPF}}
- \underbrace{f_{l1} \times f_{l2} \times (N_{FFP} - N_{FFF} \times f_{l3})}_{\textrm{interference of FPP and FFP}}\\
 + N_{PFP} \times f_{l2} -  \underbrace{N_{FFF} \times f_{11} \times f_{l2} \times f_{l3}}_{\textrm{interference of PFP and FFF}}
 - \underbrace{f_{l1} \times f_{l2} \times (N_{FFP} - N_{FFF}  \times f_{l3})}_{\textrm{interference of PFP and FFP}}\\
  - \underbrace{f_{l2} \times f_{l3} \times (N_{PFF} -  N_{FFF} \times f_{11} )}_{\textrm{interference of PFP and PFF}} = \\
  N_{FPP} \times f_{l1} +  N_{PFP} \times f_{l2} - 2 N_{FFF} \times f_{11} \times f_{l2} \times f_{l3}
   - 2 N_{FFP} \times f_{l1} \times f_{l2} \\ - N_{FPF} \times f_{l1} \times f_{l3} -  N_{PFF} \times f_{l2} \times f_{l3}
\end{multline*}

where $f_{l1} = \frac{\varepsilon_{\text{fake}}}{1 - \varepsilon_{\text{fake}}}$, 
computed from the misidentification ratio $\varepsilon_{\text{fake}}$ in the 
particular $p_{\text{T}}$ and $\eta$ bin.

Combining the yields from all seven control regions in this way yields the following formula
for the total number of estimated background events from all regions 

\begin{align*}
N_B = N_{FPP} \times FR_{l1} + N_{PFP} \times FR_{l2} + N_{PPF} \times FR_{13} 
 + N_{FFF} \times FR_{11} \times FR_{l2} \times FR_{l3}  \\
- N_{FPF} \times FR_{l1} \times FR_{l3} - N_{FFP} \times FR_{l1} \times FR_{l2} - N_{PFF} \times FR_{l2} \times FR_{l3} \,.
\label{eqn:fakerate}
\end{align*}

The characteristics of the control region events
are used to obtain per-channel differential predictions in the signal region.
The expected contribution of nonprompt events in the signal region is shown
in Fig.~\ref{fig:expectedEWSignal}, where the signal and background 
normalizations are fixed to their expected
values and the expected sum of signal and background events is treated as the 
observed data.


\begin{figure}[htbp]
  \centering
   \includegraphics[width=0.5\textwidth]{figures/AnalysisProcedure/expectedYield_EWSignal.pdf}
  \caption{
    Expected composition of signal and background events for the EW signal selection.
    The expected sum of signal and background contributions are treated as the observed
    data in a maximum likelihood fit. The uncertainty in the total expected event yield
    from WZ and non-WZ contributions is shown as a hashed band.
        }
 \label{fig:expectedEWSignal}
\end{figure}

This method is validated in nonoverlapping data samples enriched in Drell--Yan and $\ttbar$ contributions.
The Drell--Yan region is defined by inverting the selection requirement in $\ptmiss$, and
the $\ttbar$ region is defined by requiring at least one b-tagged jet and rejecting events with $|m_{\ell'\ell'} - m_\PZ| < 5\GeV$
while keeping all other requirements for the signal region.
The background prediction and observed data in these regions for events statisfying the {\WZjj} selection are shown in
Fig.~\ref{fig:nonpromptValidationRegions}, which demonstrates agreement between the two techiques.
The predictions for large values of $\mjj$ and
$\etajj$ are limited by the event count of data events used for the nonprompt esitmation.

\begin{figure}[htbp]
  \centering
   \includegraphics[width=0.4\textwidth]{figures/AnalysisProcedure/mjj_3lDYControl.pdf}
   \includegraphics[width=0.4\textwidth]{figures/AnalysisProcedure/mjj_3lDYControl_dijetFRs.pdf}
   \includegraphics[width=0.4\textwidth]{figures/AnalysisProcedure/dEtajj_3lTTbarControl.pdf}
   \includegraphics[width=0.4\textwidth]{figures/AnalysisProcedure/dEtajj_3lTTbarControl_dijetFRs.pdf}
  \caption{
    $\mjj$ (top) for events in the Drell--Yan control region (top)
    and $\etajj$ for events in the $\ttbar$ control region(bottom).
    The events are required to have two $\pt > 30\GeV$
    jets but no additional kinematic selection is applied to the dijet system.
        }
 \label{fig:fakeRates2D}
\end{figure}

The event count of the control samples of nonprompt events
satisfying the EW signal selection severely limits
differential predictions in this region. This is illustrated in 
Fig.~\ref{fig:nonpromptdEtajjByChan}, which shows the differential
prediction from the nonprompt background in the EW signal region.
The per-channel predictions are severely limited by the event count
of the control regions.

\begin{figure}[htbp]
  \centering
   \includegraphics[width=0.4\textwidth]{figures/AnalysisProcedure/dEtajj_nonprompt.pdf}
   \includegraphics[width=0.4\textwidth]{figures/AnalysisProcedure/dEtajj_nonprompt_eem.pdf}
  \caption{
    Distribution of $\etajj$ for estimated nonprompt events in the EW signal region. The
    left plot shows the combined contribution for all channels, while the right plot shows
    the contribution from only the ee$\mu$ channel, which is severally limited by the 
    control region event count. 
        }
 \label{fig:fakeRates2D}
\end{figure}

The combined shape of the nonprompt background for all channels 
is therefore used for each channel in the \EWWZ signal strength
measurement and in the extraction of constraints on anomalous quartic
gauge couplings. 
The normalization of the distribution per channel is taken from the 
ratio of the nonprompt yield in a single channel to the total nonprompt event yield 
measured in \WZjj events with no requirements on the dijet system.
These ratios are found to be consistent within the statistical uncertainty with ratios measured
when relaxing the jet \PT requirement in \WZjj events, in \WZ events inclusive in the number of jets, 
and in events satisfying the EW signal and control region selections, as demonstrated in
Table~\ref{tab:nonpromptNorms}.


\begin{table}[htbp]
     \centering
     \caption{
       Relative fraction of nonprompt events by channel for different selections of \WZ and \WZjj events.
           }
     \begin{tabular}{l|cccc}
 \hline %------------------------------------------------------------------------------------------
       Selection           &   \eee           & \eem             &   \emm         &   \mmm  \\	
 \hline %------------------------------------------------------------------------------------------
 \hline %------------------------------------------------------------------------------------------
       \WZ inclusive       & $ 0.06 \pm 0.04$	& $0.18 \pm 0.03$  & $0.21 \pm 0.04	$ & $0.55 \pm 0.04$ \\
       \WZjj               & $ 0.08 \pm 0.05$	& $0.16 \pm 0.05$  & $0.33 \pm 0.07	$ & $0.44 \pm 0.06$ \\
       Charged Higgs boson & $ 0.14 \pm 0.14$	& $0.22 \pm 0.14$  & $0.30 \pm 0.21	$ & $0.34 \pm 0.15$ \\
        EW signal          & $< 0.1 \pm 0.1 $	& $0.23 \pm 0.19$  & $0.57 \pm 0.34 $	& $0.21 \pm 0.16$ \\
  \hline
  \end{tabular}
  \label{tab:nonpromptNorms}
\end{table}

\section{Statistical procedures}
Overview of the likelihood and assymptotic formulae
\subsection{WZjj cross section measurement}
Briefly describe likelihood
\subsection{Significance of electroweak WZ Production}
Briefly describe 2D fit
\subsection{Limits on charged Higgs bosons}
Briefly describe fit to MT
\subsection{Limits on anomalous quartic gauge couplings}

\section{Systematic uncertainties}

The dominant uncertainties in both the cross section measurement 
and new physics searches are those associated with 
the jet energy scale (JES) and resolution (JER).  The JES and JER 
uncertainties are applied in simulated events by smearing and 
scaling the relevant
observables and propagating the effects to the event selection and 
the kinematic variables used in the analysis.
The uncertainty in the event yield in the EW signal selection
due to the JES and JER 
is found to be 9\% for \QCDWZ and 5\% for \EWWZ processes.
The uncertainty additionally depends on the $\pt$ and $\eta$ of the selected
jets. For the \QCDWZ process, it varies in the range of 5--25\% with
increasing values of {\mjj} and $\left|\etajj\right|$.

Uncertainties in signal and background processes estimated with
MC simulation are evaluated from the theoretical uncertainties 
of the predictions. 
Event weights in the simulated samples are used to evaluate 
variations of the central prediction.
Scale uncertainties are estimated by varying
$\mu_{\mathrm{R}}$ and $\mu_{\mathrm{F}}$ by a factor of two from their
nominal values, with the condition that $1/2 \le \mu_{\mathrm{R}}/\mu_{\mathrm{F}} \le 2$.
The maximal and minimal variations are obtained
per bin to form a shape-dependent variation band.
The PDF uncertainties are obtained from the MC replica sets 
of the NNPDF3.0 PDF. 
The PDF and scale uncertainties are uncorrelated for different signal and
background process and 100\% correlated across bins for the distributions
used to extract results.

The uncertainty in modeling the \EWWZ and \QCDWZ
processes has a large impact in the \EWWZ signal strength measurement.
In addition to the uncertainties from PDF and scale choice, 
comparisons of alternative matrix element and parton shower generators are
considered.
The uncertainty in the \QCDWZ process is derived by
comparing the predictions of the MLM-merged sample and those obtained with the \FxFx-merged sample,
after fixing the normalization in the control region described in Section~\ref{sec:backgrounds}
to the observed data yield in this region.
Differences between the predictions of the MC simulations
in the signal region and in the ratio
of the control region to the signal region event yields
are considered in the comparisons.
The differences in predictions are generally found to be
within the scale and PDF uncertainties of the samples,
and a 10\% normalization uncertainty is assigned to account for
the observed discrepancies.
The results obtained using the \POWHEG sample,
which predicts a slightly softer {\mjj} spectrum, are also largely contained 
within the theoretical uncertainties considered.
However, because \WZjj events from this sample arise from
soft radiation from the parton shower, it is 
not explicitly considered in the uncertainty evaluation.
For the \EWWZ process, the samples described in Section~\ref{sec:mc}
are found to agree within
the theoretical uncertainties from the PDF and the choice of 
$\mu_{\mathrm{R}}$ and $\mu_{\mathrm{F}}$ 
for the kinematic variables considered in the analysis,
so no additional uncertainty is
assigned. 


The interference term is evaluated on particle-level simulated events
selected from the samples described in Section~\ref{sec:mc}. 
It is found to be positive, and amounts to 12\% 
of the \EWWZ contribution in the control region and 4\% in the signal region
for both MC samples considered. 
The ratio of the interference to the \EWWZ 
decreases with increasing $\mjj$, consistent with the observations of Ref.~\cite{leshouches2017}.
These values are used as a symmetric shape uncertainty in the 
signal cross section when performing the 
\EWWZ signal strength measurement.
This uncertainty is subdominant with respect to other theoretical uncertainties 
and has a negligible contribution to the uncertainty 
on the observed \EWWZ signal strength.

Higher-order EW corrections in VBS processes are known to be negative and at
the level of tens of percent, with the correction increasing in magnitude 
with increasing {\mjj}~\cite{Biedermann:2016yds}.
While this behavior is expected to be a general feature of VBS processes, 
calculations have not yet been performed for the final state considered 
in this letter, and they are therefore not applied to the EW WZ MC simulation. 
However, with the exception of the highest values of {\mjj}, modifications 
are expected to be contained within the theoretical uncertainties and 
we have confirmed that the EW WZ signal strength measurement is 
insensitive to corrections of the shape and magnitude calculated for 
other EW production processes

Uncertainties related to the finite number of simulated events, or to the limited 
number of events in data control regions, affect the signal and background predictions. 
They are uncorrelated 
across different samples, and across bins of a single distribution. 
The limited event count in the control regions of the
nonprompt background estimate is the dominant contribution to this uncertainty.

The nonprompt background estimate is also affected by systematic uncertainties
from the jet flavor composition of the control regions and loose-to-tight transfer factors.
The systematic uncertainty in the nonprompt event yield is 30\%
for both electrons and muons, uncorrelated between channels.
It covers the largest difference observed
between the estimated and measured
numbers of events in data control samples enriched in
$\ttbar$ and Drell--Yan contributions and the differences between 
using transfer factors derived in \Zpj and dijet events.


\begin{table}[htbp]
     \centering
     \caption{ The dominant uncertainty contributions in the fiducial 
         \WZjj cross section measurement 
         and their expected contributions to the significance of the
         \EWWZ signal strength measurement. The impact of each systematic 
         uncertainty in the \WZjj 
         cross section measurement is obtained by freezing the set of associated nuisance 
         parameters to their best fit values and comparing the total uncertainty in the signal strength
         to the result from the nominal fit. 
         The effect on the \EWWZ significance, shown in the last column,
         is defined as the relative increase in the expected significance when
         freezing the nuisance term to its best fit value.
           }
     \begin{tabular}{l|ccc}
 \hline %------------------------------------------------------------------------------------------
     Source of systematic uncertainty & \multicolumn{3}{c}{Relative systematic uncertainty [\%]} \\
                                      & $\sigma_{\WZjj}$ & \EWWZ significance \\
 \hline %------------------------------------------------------------------------------------------
 \hline %------------------------------------------------------------------------------------------
 Jet energy scale                     & $+10.7\, /-8.1$ & $ 7.0 $               \\ %done
 Jet energy resolution                & $+1.9\,/-2.1$   & $< 0.1$             \\ %done
 \QCDWZ modeling                      &    N/A          & $ 2.2 $             \\
 Other background theory              &  $+2.2\,/-2.2$  & $ 0.3 $             \\ %done
 Nonprompt normalization              &  $+2.5\,/-2.5$  & $ 0.3 $             \\ %done
 Nonprompt event count                &  $+6.0\,/-5.8$  & $ 1.7 $               \\ %done
 Lepton energy scale and eff.         &  $+3.5\,/-2.7$  & $< 0.1$             \\ %done
 b tagging                            &  $+2.0\,/-1.7$  & $< 0.1$             \\ %done
 Integrated luminosity                &  $+3.6\,/-3.0$  & $< 0.1$             \\ %done
 \hline %------------------------------------------------------------------------------------------
      \end{tabular}
     \label{tab:systematics}
\end{table}

